Adverse Events (AE) are harmful events resulting from the use of medical products. Although social media may be crucial for early AE detection, the sheer scale of this data makes it logistically intractable to analyze using human agents, with NLP representing the only low-cost and scalable alternative. In this paper, we frame AE Detection and Extraction as a sequence-to-sequence problem using the T5 model architecture and achieve strong performance improvements over the baselines on several English benchmarks (F1 = 0.71, 12.7\% relative improvement for AE Detection; Strict F1 = 0.713, 12.4\% relative improvement for AE Extraction). Motivated by the strong commonalities between AE tasks, the class imbalance in AE benchmarks, and the linguistic and structural variety typical of social media texts, we propose a new strategy for multi-task training that accounts, at the same time, for task and dataset characteristics. Our approach increases model robustness, leading to further performance gains. Finally, our framework shows some language transfer capabilities, obtaining higher performance than Multilingual BERT in zero-shot learning on French data.
