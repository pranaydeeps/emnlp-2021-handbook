Within the currently dominant Minimalist framework for syntax (Chomsky, 1995, 2000), it is not uncommon to encounter multiple proposals for the same natural language pattern in the literature. We investigate the possibility of evaluating and comparing analyses of syntax phenomena, implemented as minimalist grammars (Stabler, 1997), from a quantitative point of view. This paper introduces a principled way of making linguistic generalizations by detecting and eliminating syntactic and phonological redundancies in the data. As proof of concept, we first provide a small step-by-step example transforming a naive grammar over unsegmented words into a linguistically motivated grammar over morphemes, and then discuss a description of the English auxiliary system, passives, and raising verbs produced by a prototype implementation of a procedure for automated grammar optimization.
