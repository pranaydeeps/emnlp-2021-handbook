We present a systematic study on multilingual and cross-lingual intent detection (ID) from spoken data. The study leverages a new resource put forth in this work, termed MInDS-14, a first training and evaluation resource for the ID task with spoken data. It covers 14 intents extracted from a commercial system in the e-banking domain, associated with spoken examples in 14 diverse language varieties. Our key results indicate that combining machine translation models with state-of-the-art multilingual sentence encoders (e.g., LaBSE) yield strong intent detectors in the majority of target languages covered in MInDS-14, and offer comparative analyses across different axes: e.g., translation direction, impact of speech recognition, data augmentation from a related domain. We see this work as an important step towards more inclusive development and evaluation of multilingual ID from spoken data, hopefully in a much wider spectrum of languages compared to prior work.
