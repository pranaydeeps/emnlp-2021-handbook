Defeasible reasoning is the mode of reasoning where conclusions can be overturned by taking into account new evidence. Existing cognitive science literature on defeasible reasoning suggests that a person forms a ``mental model'' of the problem scenario before answering questions. Our research goal asks whether neural models can similarly benefit from envisioning the question scenario before answering a defeasible query. Our approach is, given a question, to have a model first create a graph of relevant influences, and then leverage that graph as an additional input when answering the question. Our system, CURIOUS, achieves a new state-of-the-art on three different defeasible reasoning datasets. This result is significant as it illustrates that performance can be improved by guiding a system to ``think about'' a question and explicitly model the scenario, rather than answering reflexively.
