Mining the causes of political decision-making is an active research area in the field of political science. In the past, most studies have focused on long-term policies that are collected over several decades of time, and have primarily relied on surveys as the main source of predictors. However, the recent COVID-19 pandemic has given rise to a new political phenomenon, where  political decision-making consists of frequent short-term decisions, all on the same controlled topic---the pandemic. In this paper, we focus on the question of how public opinion influences policy decisions, while controlling for confounders such as COVID-19 case increases or unemployment rates. Using a dataset consisting of Twitter data from the 50 US states, we classify the sentiments toward governors of each state, and conduct controlled studies and comparisons.  Based on the compiled samples of sentiments, policies, and confounders, we conduct causal inference to discover trends in political decision-making across different states.
