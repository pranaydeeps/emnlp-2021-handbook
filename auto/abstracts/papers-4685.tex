Traditional hand-crafted linguistically-informed features have often been used for distinguishing between translated and original non-translated texts. By contrast, to date, neural architectures without manual feature engineering have been less explored for this task. In this work, we (i) compare the traditional feature-engineering-based approach to the feature-learning-based one and (ii) analyse the neural architectures in order to investigate how well the hand-crafted features explain the variance in the neural models' predictions. We use pre-trained neural word embeddings, as well as several end-to-end neural architectures in both monolingual and multilingual settings and compare them to feature-engineering-based SVM classifiers. We show that (i) neural architectures outperform other approaches by more than 20 accuracy points, with the BERT-based model performing the best in both the monolingual and multilingual settings; (ii) while many individual hand-crafted translationese features correlate with neural model predictions, feature importance analysis shows that the most important features for neural and classical architectures differ; and (iii) our multilingual experiments provide empirical evidence for translationese universals across languages.
