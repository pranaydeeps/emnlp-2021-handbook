In this paper, we describe the approach used by the UPF-taln team for tasks 10 and 11 of SemEval 2015 that respectively focused on ``Sentiment Analysis in Twitter'' and ``Sentiment Analysis of Figurative Language in Twitter''. Our approach achieved satisfactory results in the figurative language analysis task, obtaining the second best result. In task 10, our approach obtained acceptable performances. We experimented with both word-based features and domain-independent intrinsic word features. We exploited two machine learning methods: the supervised algorithm Support Vector Machine for task 10, and Random-Sub-Space with M5P as base algorithm for task 11.
