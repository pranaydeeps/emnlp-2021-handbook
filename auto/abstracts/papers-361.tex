This paper investigates how to correct Chinese text errors with types of mistaken, missing and redundant characters, which are common for Chinese native speakers. Most existing models based on detect-correct framework can correct mistaken characters, but cannot handle missing or redundant characters due to inconsistency between model inputs and outputs. Although Seq2Seq-based or sequence tagging methods provide solutions to the three error types and achieved relatively good results in English context, they do not perform well in Chinese context according to our experiments. In our work, we propose a novel alignment-agnostic detect-correct framework that can handle both text aligned and non-aligned situations and can serve as a cold start model when no annotation data are provided. Experimental results on three datasets demonstrate that our method is effective and achieves a better performance than most recent published models.
