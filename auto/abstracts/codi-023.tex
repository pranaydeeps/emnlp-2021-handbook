In the field of humor research, there has been a recent surge of interest in the sub-domain of Conversational Humor (CH). This study has two main objectives. (a) develop a conversational (humorous and non-humorous) dataset in Telugu. (b) detect CH in the compiled dataset. In this paper, the challenges faced while collecting the data and experiments carried out are elucidated. Transfer learning and non-transfer learning techniques are implemented by utilizing pre-trained models such as FastText word embeddings, BERT language models and Text GCN, which learns the word and document embeddings simultaneously of the corpus given. State-of-the-art results are observed with a 99.3\% accuracy and a 98.5\% f1 score achieved by BERT.
