The Multi-label Emotion Classification task aims to identify all possible emotions in a written text that best represents the author's mental state. The majority of benchmark corpora have been developed for the English language (monolingual) using tweets. However, Multi-label Emotion Classification has not been explored for code-mixed text (English and Roman Urdu), although code-mixed text like Roman Urdu (written with English alphabets) is widely used on social media, particularly by the South Asian community. This study presents a large benchmark corpus comprising 12,000 code-mixed (English and Roman Urdu) SMS messages to fulfill this gap. Each code-mixed SMS message is annotated using a set of 12 emotions, including anger, anticipation, disgust, fear, joy, love, optimism, pessimism, sadness, surprise, trust, and neutral (no emotion). We applied Content-based methods (3-word n-gram features and eight character n-gram features) and deep learning-based methods (CNN, RNN, Bi-RNN, GRU, Bi-GRU, LSTM, and Bi-LSTM). Our CNN outperformed all other methods (Micro Precision = 0.79, Micro Recall = 0.55, Micro F1 = 0.65, and Macro F1 = 0.67).
