One of the central aspects of contextualised language models is that they should be able to distinguish the meaning of lexically ambiguous words by their contexts. In this paper we investigate the extent to which the contextualised embeddings of word forms that display multiplicity of sense reflect traditional distinctions of polysemy and homonymy. To this end, we introduce an extended, human-annotated dataset of graded word sense similarity and co-predication acceptability, and evaluate how well the similarity of embeddings predicts similarity in meaning. Both types of human judgements indicate that the similarity of polysemic interpretations falls in a continuum between identity of meaning and homonymy. However, we also observe significant differences within the similarity ratings of polysemes, forming consistent patterns for different types of polysemic sense alternation. Our dataset thus appears to capture a substantial part of the complexity of lexical ambiguity, and can provide a realistic test bed for contextualised embeddings. Among the tested models, BERT Large shows the strongest correlation with the collected word sense similarity ratings, but struggles to consistently replicate the observed similarity patterns. When clustering ambiguous word forms based on their embeddings, the model displays high confidence in discerning homonyms and some types of polysemic alternations, but consistently fails for others.
