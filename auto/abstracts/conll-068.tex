The most straightforward approach to joint word segmentation (WS), part-of-speech (POS) tagging, and constituent parsing is converting a word-level tree into a char-level tree, which, however, leads to two severe challenges. First, a larger label set (e.g.,  ≥ 600) and longer inputs both increase computational costs. Second, it is difficult to rule out illegal trees containing conflicting production rules,  which is important for reliable model evaluation. If a POS tag (like VV) is above a phrase tag (like VP) in the output tree, it becomes quite complex to decide word boundaries. To deal with both challenges, this work proposes a two-stage coarse-to-fine labeling framework for joint WS-POS-PAR. In the coarse labeling stage, the joint model outputs a bracketed tree, in which each node corresponds to one of four labels (i.e., phrase, subphrase, word, subword). The tree is guaranteed to be legal via constrained CKY decoding. In the fine labeling stage, the model expands each coarse label into a final label (such as VP, VP*, VV, VV*). Experiments on Chinese Penn Treebank 5.1 and 7.0 show that our joint model consistently outperforms the pipeline approach on both settings of w/o and w/ BERT, and achieves new state-of-the-art performance.
