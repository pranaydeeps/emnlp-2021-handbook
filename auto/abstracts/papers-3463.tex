Recent work has shown that dense passage retrieval techniques achieve better ranking accuracy in open-domain question answering compared to sparse retrieval techniques such as BM25, but at the cost of large space and memory requirements. In this paper, we analyze the redundancy present in encoded dense vectors and show that the default dimension of 768 is unnecessarily large. To improve space efficiency, we propose a simple unsupervised compression pipeline that consists of principal component analysis (PCA), product quantization, and hybrid search. We further investigate other supervised baselines and find surprisingly that unsupervised PCA outperforms them in some settings. We perform extensive experiments on five question answering datasets and demonstrate that our best pipeline achieves good accuracy--space trade-offs, for example, $48\times$ compression with less than $3\%$ drop in top-$100$ retrieval accuracy on average or $96\times$ compression with less than $4\%$ drop. Code and data are available at \url{http://pyserini.io/}.
