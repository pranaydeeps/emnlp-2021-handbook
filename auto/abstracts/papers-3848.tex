Recently, kNN-MT (Khandelwal et al., 2020) has shown the promising capability of directly incorporating the pre-trained neural machine translation (NMT) model with domain-specific token-level k-nearest-neighbor (kNN) retrieval to achieve domain adaptation without retraining. Despite being conceptually attractive, it heavily relies on high-quality in-domain parallel corpora, limiting its capability on unsupervised domain adaptation, where in-domain parallel corpora are scarce or nonexistent. In this paper, we propose a novel framework that directly uses in-domain monolingual sentences in the target language to construct an effective datastore for k-nearest-neighbor retrieval. To this end, we first introduce an autoencoder task based on the target language, and then insert lightweight adapters into the original NMT model to map the token-level representation of this task to the ideal representation of the translation task. Experiments on multi-domain datasets demonstrate that our proposed approach significantly improves the translation accuracy with target-side monolingual data, while achieving comparable performance with back-translation. Our implementation is open-sourced at https://github. com/zhengxxn/UDA-KNN.
