Despite significant recent progress in coreference resolution, the quality of current state-of-the-art systems still considerably trails behind human-level performance. Using the CoNLL-2012 and PreCo datasets, we dissect the best instantiation of the mainstream end-to-end coreference resolution model that underlies most current best-performing coreference systems, and empirically analyze the behavior of its two components: mention detector and mention linker. While the detector traditionally focuses heavily on recall as a design decision, we demonstrate the importance of precision, calling for their balance. However, we point out the difficulty in building a precise detector due to its inability to make important anaphoricity decisions. We also highlight the enormous room for improving the linker and show that the rest of its errors mainly involve pronoun resolution. We propose promising next steps and hope our findings will help future research in coreference resolution.
