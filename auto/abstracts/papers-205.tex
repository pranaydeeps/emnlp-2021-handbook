Linguistic borrowing is the phenomenon of transferring linguistic constructions (lexical, phonological, morphological, and syntactic) from a ``donor'' language to a ``recipient'' language as a result of contacts between communities speaking different languages. Borrowed words are found in all languages, and—in contrast to cognate relationships—borrowing relationships may exist across unrelated languages (for example, about 40\% of Swahili's vocabulary is borrowed from Arabic). In this paper, we develop a model of morpho-phonological transformations across languages with features based on universal constraints from Optimality Theory (OT). Compared to several standard—but linguistically naïve—baselines, our OT-inspired model obtains good performance with only a few dozen training examples, making this a cost-effective strategy for sharing lexical information across languages.
