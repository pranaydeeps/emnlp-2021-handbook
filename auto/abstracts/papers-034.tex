Two goals are targeted by computer philology for ancient manuscript corpora: firstly, making an edition, that is roughly speaking one text version representing the whole corpus, which contains variety induced through copy errors and other processes and secondly, producing a stemma. A stemma is a graph-based visualization of the copy history with manuscripts as nodes and copy events as edges. Its root, the so-called archetype, is the supposed original text or urtext from which all subsequent copies are made. Our main contribution is to present one of the first computational approaches to automatic archetype reconstruction and to introduce the first text-based evaluation for automatically produced archetypes. We compare a philologically generated archetype with one generated by bio-informatic software.
