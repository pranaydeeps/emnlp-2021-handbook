Natural Language Generation (NLG) for task-oriented dialogue systems focuses on communicating specific content accurately, fluently, and coherently. While these attributes are crucial for a successful dialogue, it is also desirable to simultaneously accomplish specific stylistic goals, such as response length, point-of-view, descriptiveness, sentiment, formality, and empathy. In this work, we focus on stylistic control and evaluation for schema-guided NLG, with joint goals of achieving both semantic and stylistic control. We experiment in detail with various controlled generation methods for large pretrained language models: specifically, conditional training, guided fine-tuning, and guided decoding. We discuss their advantages and limitations, and evaluate them with a broad range of automatic and human evaluation metrics. Our results show that while high style accuracy and semantic correctness are easier to achieve for more lexically-defined styles with conditional training, stylistic control is also achievable for more semantically complex styles using discriminator-based guided decoding methods. The results also suggest that methods that are more scalable (with less hyper-parameters tuning) and that disentangle context generation and stylistic variations are more effective at achieving semantic correctness and style accuracy.
