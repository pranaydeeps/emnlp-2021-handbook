Computational resources such as semantically annotated corpora can play an important role in enabling speakers of indigenous minority languages to participate in government, education, and other domains of public life in their own language. However, many languages -- mainly those with small native speaker populations and without written traditions -- have little to no digital support. One hurdle in creating such resources is that for many languages, few speakers would be capable of annotating texts -- a task which requires literacy and some linguistic training -- and that these experts' time is typically in high demand for language planning work. This paper assesses whether typologically trained non-speakers of an indigenous language can feasibly perform semantic annotation using Uniform Meaning Representations, thus allowing for the creation of computational materials without putting further strain on community resources.
