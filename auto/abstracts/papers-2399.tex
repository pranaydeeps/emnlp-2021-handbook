State-of-the-art multilingual systems rely on shared vocabularies that sufficiently cover all considered languages.  To this end, a  simple and frequently used approach makes use of subword vocabularies constructed jointly over several languages. We hypothesize that such vocabularies are suboptimal due to false positives (identical subwords with different meanings across languages) and false negatives (different subwords with similar meanings).  To address these issues, we propose Subword Mapping and Anchoring across Languages (SMALA), a method to construct bilingual subword vocabularies. SMALA extracts subword alignments using an unsupervised state-of-the-art mapping technique and uses them to create cross-lingual anchors based on subword similarities. We demonstrate the benefits of SMALA for cross-lingual natural language inference (XNLI), where it improves zero-shot transfer to an unseen language without task-specific data, but only by sharing subword embeddings.  Moreover, in neural machine translation, we show that joint subword vocabularies obtained with SMALA lead to higher BLEU scores on sentences that contain many false positives and false negatives.
