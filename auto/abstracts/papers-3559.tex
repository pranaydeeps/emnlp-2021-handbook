Many NLG tasks such as summarization, dialogue response, or open domain question answering, focus primarily on a source text in order to generate a target response. This standard approach falls short, however, when a user's intent or context of work is not easily recoverable based solely on that source text-- a scenario that we argue is more of the rule than the exception. In this work, we argue that NLG systems in general should place a much higher level of emphasis on making use of additional context, and suggest that relevance (as used in Information Retrieval) be thought of as a crucial tool for designing user-oriented text-generating tasks. We further discuss possible harms and hazards around such personalization, and argue that value-sensitive design represents a crucial path forward through these challenges.
