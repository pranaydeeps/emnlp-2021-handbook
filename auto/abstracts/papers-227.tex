Large-scale multi-label text classification (LMTC) tasks often face long-tailed label distributions, where many labels have few or even no training instances. Although current methods can exploit prior knowledge to handle these few/zero-shot labels, they neglect the meta-knowledge contained in the dataset that can guide models to learn with few samples. In this paper, for the first time, this problem is addressed from a meta-learning perspective. However, the simple extension of meta-learning approaches to multi-label classification is sub-optimal for LMTC tasks due to long-tailed label distribution and coexisting of few- and zero-shot scenarios. We propose a meta-learning approach named META-LMTC. Specifically, it constructs more faithful and more diverse tasks according to well-designed sampling strategies and directly incorporates the objective of adapting to new low-resource tasks into the meta-learning phase. Extensive experiments show that META-LMTC achieves state-of-the-art performance against strong baselines and can still enhance powerful BERTlike models.
