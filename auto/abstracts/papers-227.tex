Several techniques have recently been pro- posed for training ``self-normalized'' discriminative models. These attempt to find parameter settings for which unnormalized model scores approximate the true label probability. However, the theoretical properties of such techniques (and of self-normalization generally) have not been investigated. This paper examines the conditions under which we can expect self-normalization to work. We characterize a general class of distributions that admit self-normalization, and prove generalization bounds for procedures that minimize empirical normalizer variance. Motivated by these results, we describe a novel variant of an established procedure for training self-normalized models. The new procedure avoids computing normalizers for most training examples, and decreases training time by as much as factor of ten while preserving model quality.
