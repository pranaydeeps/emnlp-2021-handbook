Humans are capable of learning novel concepts from very few examples; in contrast, state-of-the-art machine learning algorithms typically need thousands of examples to do so. In this paper, we propose an algorithm for learning novel concepts by representing them as programs over existing concepts. This way the concept learning problem is naturally a program synthesis problem and our algorithm learns from a few examples to synthesize a program representing the novel concept. In addition, we perform a theoretical analysis of our approach for the case where the program defining the novel concept over existing ones is context-free. We show that given a learned grammar-based parser and a novel production rule, we can augment the parser with the production rule in a way that provably generalizes. We evaluate our approach by learning concepts in the semantic parsing domain extended to the few-shot novel concept learning setting, showing that our approach significantly outperforms end-to-end neural semantic parsers.
