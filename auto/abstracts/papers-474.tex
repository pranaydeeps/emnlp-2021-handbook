With the early success of query-answer assistants such as Alexa and Siri, research attempts to expand system capabilities of handling service automation are now abundant. However, preliminary systems have quickly found the inadequacy in relying on simple classification techniques to effectively accomplish the automation task. The main challenge is that the dialogue often involves complexity in user's intents (or purposes) which are multiproned, subject to spontaneous change, and difficult to track. Furthermore, public datasets have not considered these complications and the general semantic annotations are lacking which may result in zero-shot problem. Motivated by the above, we propose a Label-Aware BERT Attention Network (LABAN) for zero-shot multi-intent detection. We first encode input utterances with BERT and construct a label embedded space by considering embedded semantics in intent labels. An input utterance is then classified based on its projection weights on each intent embedding in this embedded space. We show that it successfully extends to few/zero-shot setting where part of intent labels are unseen in training data, by also taking account of semantics in these unseen intent labels. Experimental results show that our approach is capable of detecting many unseen intent labels correctly.  It also achieves the state-of-the-art performance on five multi-intent datasets in normal cases.
