We describe how a question-answering system can learn about its domain from conversational dialogs. Our system learns to relate concepts in science questions to propositions in a fact corpus, stores new concepts and relations in a knowledge graph (KG), and uses the graph to solve questions. We are the first to acquire knowledge for question-answering from open, natural language dialogs without a fixed ontology or domain model that predetermines what users can say. Our relation-based strategies complete more successful dialogs than a query expansion baseline, our task-driven relations are more effective for solving science questions than relations from general knowledge sources, and our method is practical enough to generalize to other domains.
