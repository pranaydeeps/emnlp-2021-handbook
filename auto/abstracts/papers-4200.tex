Question Answering (QA) tasks requiring information from multiple documents often rely on a retrieval model to identify relevant information for reasoning. The retrieval model is typically trained to maximize the likelihood of the labeled supporting evidence. However, when retrieving from large text corpora such as Wikipedia, the correct answer can often be obtained from multiple evidence candidates. Moreover, not all such candidates are labeled as positive during annotation, rendering the training signal weak and noisy. This problem is exacerbated when the questions are unanswerable or when the answers are Boolean, since the model cannot rely on lexical overlap to make a connection between the answer and supporting evidence. We develop a new parameterization of set-valued retrieval that handles unanswerable queries, and we show that marginalizing over this set during training allows a model to mitigate false negatives in supporting evidence annotations. We test our method on two multi-document QA datasets, IIRC and HotpotQA. On IIRC, we show that joint modeling with marginalization improves model performance by 5.5 F1 points and achieves a new state-of-the-art performance of 50.5 F1. We also show that retrieval marginalization results in 4.1 QA F1 improvement over a non-marginalized baseline on HotpotQA in the fullwiki setting.
