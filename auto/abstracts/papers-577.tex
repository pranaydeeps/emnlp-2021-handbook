Human expertise and the participation of speech communities are essential factors in the success of technologies for low-resource languages. Accordingly, we propose a new computational task which is tuned to the available knowledge and interests in an Indigenous community, and which supports the construction of high quality texts and lexicons. The task is illustrated for Kunwinjku, a morphologically-complex Australian language. We combine a finite state implementation of a published grammar with a partial lexicon, and apply this to a noisy phone representation of the signal. We locate known lexemes in the signal and use the morphological transducer to build these out into hypothetical, morphologically-complex words for human validation. We show that applying a single iteration of this method results in a relative transcription density gain of 17\%. Further, we find that 75\% of breath groups in the test set receive at least one correct partial or full-word suggestion.
