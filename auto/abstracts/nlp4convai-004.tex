Zhang et al. (2020) proposed to formulate few-shot intent classification as natural language inference (NLI) between query utterances and examples in the training set. The method is known as discriminative nearest neighbor classification or DNNC. Inspired by this work, we propose to simplify the NLI-style classification pipeline to be the entailment prediction on the utterance-semantic-label-pair (USLP). The semantic information in the labels can thus been infused into the classification process. Compared with DNNC, our proposed method is more efficient in both training and serving since it is based upon the entailment between query utterance and labels instead of all the training examples. The DNNC method requires more than one example per intent while the USLP approach does not have such constraint. In the 1-shot experiments on the CLINC150 (Larson et al., 2019) dataset, the USLP method outperforms traditional classification approach by >20 points (in-domain ac- curacy). We also find that longer and semantically meaningful labels tend to benefit model performance, however, the benefit shrinks as more training data is available.
