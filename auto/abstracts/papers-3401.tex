Basic-level categories (BLC) are an important psycholinguistic concept introduced by Rosch et al. (1976); they are defined as the most inclusive categories for which a concrete mental image of the category as a whole can be formed, and also as those categories which are acquired early in life. Rosch's original algorithm for detecting BLC (called cue-validity) is based on the availability of semantic features such as ``has tail'' for ``cat'', and has remained untested at large. An at-scale algorithm for the automatic determination of BLC exists, but it operates without Rosch-style semantic features, and is thus unable to verify Rosch's hypothesis. We present the first method for the detection of BLC at scale that makes use of Rosch-style semantic features. For both English and Mandarin, we test three methods of generating such features for any synset within Wordnet (WN): extraction of textual features from Wikipedia pages, Distributional Memory (DM) and BART. The best of our methods outperforms the current SoA in BLC detection, with an accuracy of English BLC detection of 75.0\%, and of Mandarin BLC detection 80.7\% on a test set. When applied to all of WordNet, our model predicts that 1,118 synsets in English Wordnet (1.4\%) are BLC, far fewer than existing methods, and with a precision improvement of over 200\% over these. As well as confirming the usefulness of Rosch's cue validity algorithm, we also developed and evaluated our own new indicator for BLC, which models the fact that BLC features tend to be BLC themselves.
