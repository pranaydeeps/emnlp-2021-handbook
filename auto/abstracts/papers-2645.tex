Large pre-trained language models (LMs) such as GPT-3 have acquired a surprising ability to perform zero-shot learning. For example, to classify sentiment without any training examples, we can ``prompt'' the LM with the review and the label description ``Does the user like this movie?'', and ask whether the next word is ``yes'' or ``no''. However, the next word prediction training objective is still misaligned with the target zero-shot learning objective. To address this weakness, we propose meta-tuning, which directly optimizes the zero-shot learning objective by fine-tuning pre-trained language models on a collection of datasets. We focus on classification tasks, and construct the meta-dataset by aggregating 43 existing datasets and annotating 441 label descriptions in a question-answering (QA) format. When evaluated on unseen tasks, meta-tuned models outperform a same-sized QA model and the previous SOTA zero-shot learning system based on natural language inference. Additionally, increasing parameter count from 220M to 770M improves AUC-ROC scores by 6.3\%, and we forecast that even larger models would perform better. Therefore, measuring zero-shot learning performance on language models out-of-the-box might underestimate their true potential, and community-wide efforts on aggregating datasets and unifying their formats can help build models that answer prompts better.
