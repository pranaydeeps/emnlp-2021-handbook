We explore the link between the extent to which syntactic relations are preserved in translation and the ease of correctly constructing a parse tree in a zero-shot setting. While previous work suggests such a relation, it tends to focus on the macro level and not on the level of individual edges---a gap we aim to address. As a test case, we take the transfer of Universal Dependencies (UD) parsing from English to a diverse set of languages and conduct two sets of experiments. In one, we analyze zero-shot performance based on the extent to which English source edges are preserved in translation. In another, we apply three linguistically motivated transformations to UD, creating more cross-lingually stable versions of it, and assess their zero-shot parsability. In order to compare parsing performance across different schemes, we perform extrinsic evaluation on the downstream task of cross-lingual relation extraction (RE) using a subset of a standard English RE benchmark translated to Russian and Korean. In both sets of experiments, our results suggest a strong relation between cross-lingual stability and zero-shot parsing performance.
