While powerful pre-trained language models have improved the fluency of text generation models, semantic adequacy -the ability to generate text that is semantically faithful to the input- remains an unsolved issue. In this paper, we introduce a novel automatic evaluation metric, Entity-Based Semantic Adequacy, which can be used to assess to what extent generation models that verbalise RDF (Resource Description Framework) graphs produce text that contains mentions of the  entities occurring in the RDF input. This is important as RDF subject and object entities make up 2/3 of the input. We use our metric to compare 25 models from the WebNLG Shared Tasks and we examine correlation with results from human evaluations of semantic adequacy. We show that while our metric correlates with human evaluation scores, this correlation varies with the specifics of the human evaluation setup. This suggests that in order to measure the entity-based adequacy of generated texts, an automatic metric such as the one proposed here might be more reliable, as less subjective and more focused on correct verbalisation of the input, than human evaluation measures.
