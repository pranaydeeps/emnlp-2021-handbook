Understanding when a text snippet does not provide a sought after information is an essential part of natural language utnderstanding. Recent work (SQuAD 2.0; Rajpurkar et al., 2018) has attempted to make some progress in this direction by enriching the SQuAD dataset for the Extractive QA task with unanswerable questions. However, as we show, the performance of a top system trained on SQuAD 2.0 drops considerably in out-of-domain scenarios, limiting its use in practical situations. In order to study this we build an out-of-domain corpus, focusing on simple event-based questions and distinguish between two types of IDK questions: competitive questions, where the context includes an entity of the same type as the expected answer, and simpler, non-competitive questions where there is no entity of the same type in the context. We find that SQuAD 2.0-based models fail even in the case of the simpler questions. We then analyze the similarities and differences between the IDK phenomenon in Extractive QA and the Recognizing Textual Entailments task (RTE; Dagan et al., 2013) and investigate the extent to which the latter can be used to improve the performance.
