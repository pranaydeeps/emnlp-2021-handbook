Among the most critical limitations of deep learning NLP models are their lack of interpretability, and their reliance on spurious correlations. Prior work proposed various approaches to interpreting the black-box models to unveil the spurious correlations, but the research was primarily used in human-computer interaction scenarios. It still remains underexplored whether or how such model interpretations can be used to automatically ``unlearn'' confounding features. In this work, we propose influence tuning---a procedure that leverages model interpretations to update the model parameters towards a plausible interpretation (rather than an interpretation that relies on spurious patterns in the data) in addition to learning to predict the task labels. We show that in a controlled setup, influence tuning can help deconfounding the model from spurious patterns in data, significantly outperforming baseline methods that use adversarial training.
