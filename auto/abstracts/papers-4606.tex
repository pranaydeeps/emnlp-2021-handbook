Probing experiments investigate the extent to which neural representations make properties—like part-of-speech—predictable. One suggests that a representation encodes a property if probing that representation produces higher accuracy than probing a baseline representation like non-contextual word embeddings. Instead of using baselines as a point of comparison, we're interested in measuring information that is contained in the representation but not in the baseline. For example, current methods can detect when a representation is more useful than the word identity (a baseline) for predicting part-of-speech; however, they cannot detect when the representation is predictive of just the aspects of part-of-speech not explainable by the word identity. In this work, we extend a theory of usable information called V-information and propose conditional probing, which explicitly conditions on the information in the baseline. In a case study, we find that after conditioning on non-contextual word embeddings, properties like part-of-speech are accessible at deeper layers of a network than previously thought.
