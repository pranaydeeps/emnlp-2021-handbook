Representations from large pretrained models such as BERT encode a range of features into monolithic vectors, affording strong predictive accuracy across a range of downstream tasks. In this paper we explore whether it is possible to learn disentangled representations by identifying existing subnetworks within pretrained models that encode distinct, complementary aspects. Concretely, we learn binary masks over transformer weights or hidden units to uncover subsets of features that correlate with a specific factor of variation; this eliminates the need to train a disentangled model from scratch for a particular task. We evaluate this method with respect to its ability to disentangle representations of sentiment from genre in movie reviews, toxicity from dialect in Tweets, and syntax from semantics. By combining masking with magnitude pruning we find that we can identify sparse subnetworks within BERT that strongly encode particular aspects (e.g., semantics) while only weakly encoding others (e.g., syntax). Moreover, despite only learning masks, disentanglement-via-masking performs as well as — and often better than —previously proposed methods based on variational autoencoders and adversarial training.
