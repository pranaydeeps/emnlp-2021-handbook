Question Answering (QA) has been successfully applied in scenarios of human-computer interaction such as chatbots and search engines. However, for the specific biomedical domain, QA systems are still immature due to expert-annotated datasets being limited by category and scale. In this paper, we present MLEC-QA, the largest-scale Chinese multi-choice biomedical QA dataset, collected from the National Medical Licensing Examination in China. The dataset is composed of five subsets with 136,236 biomedical multi-choice questions with extra materials (images or tables) annotated by human experts, and first covers the following biomedical sub-fields: Clinic, Stomatology, Public Health, Traditional Chinese Medicine, and Traditional Chinese Medicine Combined with Western Medicine. We implement eight representative control methods and open-domain QA methods as baselines. Experimental results demonstrate that even the current best model can only achieve accuracies between 40\% to 55\% on five subsets, especially performing poorly on questions that require sophisticated reasoning ability. We hope the release of the MLEC-QA dataset can serve as a valuable resource for research and evaluation in open-domain QA, and also make advances for biomedical QA systems.
