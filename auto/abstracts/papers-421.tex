Sentiment analysis has undergone a shift from document-level analysis, where labels express the sentiment of a whole document or whole sentence, to subsentential approaches, which assess the contribution of individual phrases, in particular including the composition of sentiment terms and phrases such as negators and intensifiers. Starting from a small sentiment treebank modeled after the Stanford Sentiment Treebank of Socher et al. (2013), we investigate suitable methods to perform compositional sentiment classification for German in a data-scarce setting, harnessing cross-lingual methods as well as existing general-domain lexical resources.
