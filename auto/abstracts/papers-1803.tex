Intent classification (IC) and slot filling (SF) are critical building blocks in task-oriented dialogue systems. These two tasks are closely-related and can flourish each other. Since only a few utterances can be utilized for identifying fast-emerging new intents and slots, data scarcity issue often occurs when implementing IC and SF. However, few IC/SF models perform well when the number of training samples per class is quite small. In this paper, we propose a novel explicit-joint and supervised-contrastive learning framework for few-shot intent classification and slot filling. Its highlights are as follows. (i) The model extracts intent and slot representations via bidirectional interactions, and extends prototypical network to achieve explicit-joint learning, which guarantees that IC and SF tasks can mutually reinforce each other. (ii) The model integrates with supervised contrastive learning, which ensures that samples from same class are pulled together and samples from different classes are pushed apart. In addition, the model follows a not common but practical way to construct the episode, which gets rid of the traditional setting with fixed way and shot, and allows for unbalanced datasets. Extensive experiments on three public datasets show that our model can achieve promising performance.
