Dialogue disentanglement aims to group utterances in a long and multi-participant dialogue into threads. This is useful for discourse analysis and downstream applications such as dialogue response selection, where it can be the first step to construct a clean context/response set. Unfortunately, labeling all{\textasciitilde}\emph{reply-to} links takes quadratic effort w.r.t the number of utterances: an annotator must check all preceding utterances to identify the one to which the current utterance is a reply. In this paper, we are the first to propose a{\textasciitilde}\textbf{zero-shot} dialogue disentanglement solution. Firstly, we train a model on a multi-participant response selection dataset harvested from the web which is not annotated; we then apply the trained model to perform zero-shot dialogue disentanglement. Without any labeled data, our model can achieve a cluster F1 score of 25. We also fine-tune the model using various amounts of labeled data. Experiments show that with only 10\% of the data, we achieve nearly the same performance of using the full dataset. Code is released at \url{https://github.com/chijames}.
