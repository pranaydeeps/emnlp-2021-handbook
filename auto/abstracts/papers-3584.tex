Existing conversational systems are mostly agent-centric, which assumes the user utterances will closely follow the system ontology. However, in real-world scenarios, it is highly desirable that users can speak freely and naturally. In this work, we attempt to build a user-centric dialogue system for conversational recommendation. As there is no clean mapping for a user's free form utterance to an ontology, we first model the user preferences as estimated distributions over the system ontology and map the user's utterances to such distributions. Learning such a mapping poses new challenges on reasoning over various types of knowledge, ranging from factoid knowledge, commonsense knowledge to the users' own situations. To this end, we build a new dataset named NUANCED that focuses on such realistic settings, with 5.1k dialogues, 26k turns of high-quality user responses. We conduct experiments, showing both the usefulness and challenges of our problem setting. We believe NUANCED can serve as a valuable resource to push existing research from the agent-centric system to the user-centric system. The code and data are publicly available.
