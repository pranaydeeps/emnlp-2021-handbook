Recent years have seen increased interest in text normalization in social media, as the informal writing styles found in Twitter and other social media data often cause problems for NLP applications. Unfortunately, most current approaches narrowly regard the normalization task as a ``one size fits all'' task of replacing non-standard words with their standard counterparts. In this work we build a taxonomy of normalization edits and present a study of normalization to examine its effect on three different downstream applications (dependency parsing, named entity recognition, and text-to-speech synthesis). The results suggest that how the normalization task should be viewed is highly dependent on the targeted application.                    The results also show that normalization must be thought of as more than word replacement in order to produce results comparable to those seen on clean text.
