Open-domain Question Answering models which directly leverage question-answer (QA) pairs, such as closed-book QA (CBQA) models and QA-pair retrievers, show promise in terms of speed and memory compared to conventional models which retrieve and read from text corpora. QA-pair retrievers also offer interpretable answers, a high degree of control, and are trivial to update at test time with new knowledge. However, these models fall short of the accuracy of retrieve-and-read systems, as substantially less knowledge is covered by the available QA-pairs relative to text corpora like Wikipedia. To facilitate improved QA-pair models, we introduce Probably Asked Questions (PAQ), a very large resource of 65M automatically-generated QA-pairs. We introduce a new QA-pair retriever, RePAQ, to complement PAQ. We find that PAQ pre-empts and caches test questions, enabling RePAQ to match the accuracy of recent retrieve-and-read models, whilst being significantly faster. Using PAQ, we train CBQA models which outperform comparable baselines by 5\%, but trail RePAQ by over 15\%, indicating the effectiveness of explicit retrieval. RePAQ can be configured for size (under 500MB) or speed (over 1K questions per second) whilst retaining high accuracy. Lastly, we demonstrate RePAQ's strength at selective QA, abstaining from answering when it is likely to be incorrect. This enables RePAQ to "back-off" to a more expensive state-of-the-art model, leading to a combined system which is both more accurate and 2x faster than the state-of-the-art model alone.