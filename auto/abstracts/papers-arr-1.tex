Many crowdsourced NLP datasets contain systematic artifacts that are identified only after data collection is complete. Earlier identification of these issues should make it easier to create high-quality training and evaluation data. We attempt this by evaluating protocols in which expert linguists work ‘in the loop’ during data collection to identify and address these issues by adjusting task instructions and incentives. Using natural language inference as a test case, we compare three data collection protocols: (i) a baseline protocol with no linguist involvement, (ii) a linguist-in-the-loop intervention with iteratively-updated constraints on the writing task, and (iii) an extension that adds direct interaction between linguists and crowdworkers via a chatroom. We find that linguist involvement does not lead to increased accuracy on out-of-domain test sets compared to baseline, and adding a chatroom has no effect on the data. Linguist involvement does, however, lead to more challenging evaluation data and higher accuracy on some challenge sets, demonstrating the benefits of integrating expert analysis during data collection.