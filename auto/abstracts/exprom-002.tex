In this paper, we present a corpus study investigating the use of the fillers äh (uh) and ähm (uhm) in informal spoken German youth language and in written text from social media. Our study shows that filled pauses occur in both corpora as markers of hesitations, corrections, repetitions and unfinished sentences, and that the form as well as the type of the fillers are distributed similarly in both registers. We present an analysis of fillers in written microblogs, illustrating that äh and ähm are used intentionally and can add a subtext to the message that is understandable to both author and reader. We thus argue that filled pauses in user-generated content from social media are words with extra-propositional meaning.
