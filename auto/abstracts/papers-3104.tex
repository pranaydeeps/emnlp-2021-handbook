Simultaneous translation is vastly different from full-sentence translation, in the sense that it starts translation before the source sentence ends, with only a few words delay. However, due to the lack of large-scale, high-quality simultaneous translation datasets, most such systems are still trained on conventional full-sentence bitexts. This is far from ideal for the simultaneous scenario due to the abundance of unnecessary long-distance reorderings in those bitexts. We propose a novel method that rewrites the target side of existing full-sentence corpora into simultaneous-style translation. Experiments on Zh$\rightarrow$En and Ja$\rightarrow$En simultaneous translation show substantial improvements (up to +2.7 BLEU) with the addition of these generated pseudo-references.
