Multilingual Named Entity Recognition (NER) is a key intermediate task which is needed in many areas of NLP. In this paper, we address the well-known issue of data scarcity in NER, especially relevant when moving to a multilingual scenario, and go beyond current approaches to the creation of multilingual silver data for the task. We exploit the texts of Wikipedia and introduce a new methodology based on the effective combination of knowledge-based approaches and neural models, together with a novel domain adaptation technique, to produce high-quality training corpora for NER. We evaluate our datasets extensively on standard benchmarks for NER, yielding substantial improvements up to 6 span-based F1-score points over previous state-of-the-art systems for data creation.
