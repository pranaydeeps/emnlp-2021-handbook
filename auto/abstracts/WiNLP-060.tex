Although no single standard for prosody labeling in Hindi exists,  researchers have employed perceptual and statistical methods in literature to draw inferences about the behavior of prosody patterns in Hindi.  This study attempts to develop a manually annotated prosody-labeled corpus of Hindi speech data of 500 sentences, based on existing research on prosodic prominence and phrasing theories. The manually transcribed sentences are being trained using pre-existing models of AuToBI to extend labeling prediction to 5,000 utterances, which will be useful for training speech models in the future.
