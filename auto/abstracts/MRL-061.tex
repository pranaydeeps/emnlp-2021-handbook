Predicting user intent and detecting the corresponding slots from text are two key problems in Natural Language Understanding (NLU). Since annotated datasets are only available for a handful of languages, our work focuses particularly on a zero-shot scenario where the target language is unseen during training. In the context of zero-shot learning, this task is typically approached using representations from pre-trained multilingual language models such as mBERT or by fine-tuning on data automatically translated into the target language. We propose a novel method which augments monolingual source data using multilingual code-switching via random translations, to enhance generalizability of large multilingual language models when fine-tuning them for downstream tasks. Experiments on the MultiATIS++ benchmark show that our method leads to an average improvement of +4.2\% in accuracy for the intent task and +1.8\% in F1 for the slot-filling task over the state-of-the-art across 8 typologically diverse languages. We also study the impact of code-switching into different families of languages on downstream performance. Furthermore, we present an application of our method for crisis informatics using a new human-annotated tweet dataset of slot filling in English and Haitian Creole, collected during the Haiti earthquake.
