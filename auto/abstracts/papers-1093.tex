Distantly supervised relation extraction (RE) automatically aligns unstructured text with relation instances in a knowledge base (KB). Due to the incompleteness of current KBs, sentences implying certain relations may be annotated as N/A instances, which causes the so-called false negative (FN) problem. Current RE methods usually overlook this problem, inducing improper biases in both training and testing procedures. To address this issue, we propose a two-stage approach. First, it finds out possible FN samples by heuristically leveraging the memory mechanism of deep neural networks. Then, it aligns those unlabeled data with the training data into a unified feature space by adversarial training to assign pseudo labels and further utilize the information contained in them. Experiments on two wildly-used benchmark datasets demonstrate the effectiveness of our approach.
