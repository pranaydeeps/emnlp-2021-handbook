Argumentative structure prediction aims to establish links between textual units and label the relationship between them, forming a structured representation for a given input text. The former task, linking, has been identified by earlier works as particularly challenging, as it requires finding the most appropriate structure out of a very large search space of possible link combinations. In this paper, we improve a state-of-the-art linking model by using multi-task and multi-corpora training strategies. Our auxiliary tasks help the model to learn the role of each sentence in the argumentative structure. Combining multi-corpora training with a selective sampling strategy increases the training data size while ensuring that the model still learns the desired target distribution well. Experiments on essays written by English-as-a-foreign-language learners show that both strategies significantly improve the model's performance; for instance, we observe a 15.8\% increase in the F1-macro for individual link predictions.
