Warning: This manuscript contains a certain level of offensive expression.
As communication through social media platforms has grown immensely, the increasing prevalence of offensive language online has become a critical problem. Notably in Korea, one of the countries with the highest Internet usage, automatic detection of offensive expressions has recently been brought to attention. However, morphological richness and complex syntax of Korean causes difficulties in neural model training. Furthermore, most of previous studies mainly focus on the detection of abusive language, disregarding implicit offensiveness and underestimating a different degree of intensity.
To tackle these problems, we present KOAS, a system that fully exploits both contextual and linguistic features and estimates an offensiveness score for a text. We carefully designed KOAS with a multi-task learning framework and constructed a Korean dataset for offensive analysis from various domains. Refer for a detailed demonstration.