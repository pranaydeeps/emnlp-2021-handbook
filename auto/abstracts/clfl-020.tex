Burrows's Delta is the most established measure for stylometric difference in literary authorship attribution. Several improvements on the original Delta have been proposed. However, a recent empirical study showed that none of the proposed variants constitute a major improvement in terms of authorship attribution performance. With this paper, we try to improve our understanding of how and why these text distance measures work for authorship attribution. We evaluate the effects of standardization and vector normalization on the statistical distributions of features and the resulting text clustering quality. Furthermore, we explore supervised selection of discriminant words as a procedure for further improving authorship attribution.
