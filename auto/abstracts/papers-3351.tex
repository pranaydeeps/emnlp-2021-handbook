Extractive summarization has been the mainstay of automatic summarization for decades. Despite all the progress, extractive summarizers still suffer from shortcomings including coreference issues arising from extracting sentences away from their original context in the source document. This affects the coherence and readability of extractive summaries. In this work, we propose a lightweight post-editing step for extractive summaries that centers around a single linguistic decision: the definiteness of noun phrases. We conduct human evaluation studies that show that human expert judges substantially prefer the output of our proposed system over the original summaries. Moreover, based on an automatic evaluation study, we provide evidence for our system's ability to generate linguistic decisions that lead to improved extractive summaries. We also draw insights about how the automatic system is exploiting some local cues related to the writing style of the main article texts or summary texts to make the decisions, rather than reasoning about the contexts pragmatically.
