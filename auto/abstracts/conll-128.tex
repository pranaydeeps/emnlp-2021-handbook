Speakers are thought to use rational information transmission strategies for efficient communication (Genzel and Charniak, 2002; Aylett and Turk, 2004; Jaeger and Levy, 2007). Previous work analysing these strategies in sentence production has failed to take into account how the information content of sentences varies as a function of the available discourse context. In this study, we estimate sentence information content within discourse context. We find that speakers transmit information at a stable rate---i.e., rationally---in English newspaper articles but that this rate decreases in spoken open domain and written task-oriented dialogues. We also observe that speakers' choices are not oriented towards local uniformity of information, which is another hypothesised rational strategy. We suggest that a more faithful model of communication should explicitly include production costs and goal-oriented rewards.
