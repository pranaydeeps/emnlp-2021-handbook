Semantic role labeling has become a key module for many language processing applications such as question answering, information extraction, sentiment analysis, and machine translation. To build an unrestricted semantic role labeler, the first step is to develop a comprehensive proposition bank. However, creating such a bank is a costly enterprise, which has only been achieved for a handful of languages. In this paper, we describe a technique to build proposition banks for new languages using distant supervision. Starting from PropBank in English and loosely parallel corpora such as versions of Wikipedia in different languages, we carried out a mapping of semantic propositions we extracted from English to syntactic structures in Swedish using named entities. We trained a semantic parser on the generated Swedish propositions and we report the results we obtained. Using the CoNLL 2009 evaluation script, we could reach the scores of 52.25 for labeled propositions and 62.44 for the unlabeled ones. We believe our approach can be applied to train semantic role labelers for other resource-scarce languages.
