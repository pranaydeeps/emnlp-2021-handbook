Although many studies use the LIWC lexicon to show the existence of verbal leakage cues in lie detection datasets, none mention how verbal leakage cues are influenced by means of data collection, or the impact thereof on the performance of models. In this paper, we study verbal leakage cues to understand the effect of the data construction method on their significance, and examine the relationship between such cues and models' validity. The LIWC word-category dominance scores of seven lie detection datasets are used to show that audio statements and lie-based annotations indicate a greater number of strong verbal leakage cue categories. Moreover, we evaluate the validity of state-of-the-art lie detection models with cross- and in-dataset testing. Results show that in both types of testing, models trained on a dataset with more strong verbal leakage cue categories---as opposed to only a greater number of strong cues---yield superior results, suggesting that verbal leakage cues are a key factor for selecting lie detection datasets.
