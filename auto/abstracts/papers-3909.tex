This paper presents an empirical study to efficiently build named entity recognition (NER) systems when a small amount of in-domain labeled data is available. Based upon recent Transformer-based self-supervised pre-trained language models (PLMs), we investigate three orthogonal schemes to improve model generalization ability in few-shot settings: (1) meta-learning to construct prototypes for different entity types, (2) task-specific supervised pre-training on noisy web data to extract entity-related representations and (3) self-training to leverage unlabeled in-domain data. On 10 public NER datasets, we perform extensive empirical comparisons over the proposed schemes and their combinations with various proportions of labeled data, our experiments show that (i)in the few-shot learning setting, the proposed NER schemes significantly improve or outperform the commonly used baseline, a PLM-based linear classifier fine-tuned using domain labels. (ii) We create new state-of-the-art results on both few-shot and training-free settings compared with existing methods.
