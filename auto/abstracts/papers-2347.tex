Neural models for the various flavours of morphological reinflection tasks have proven to be extremely accurate given ample labeled data, yet labeled data may be slow and costly to obtain. In this work we aim to overcome this annotation bottleneck by bootstrapping labeled data from a seed as small as {\em five} labeled inflection tables, accompanied by a large bulk of unlabeled text. Our bootstrapping method exploits the orthographic and semantic regularities in morphological systems in a two-phased setup, where word tagging based on {\em analogies} is followed by word pairing based on {\em distances}. Our experiments with the Paradigm Cell Filling Problem over eight typologically different languages show that in languages with relatively simple morphology, orthographic regularities on their own allow inflection models to achieve respectable accuracy. Combined orthographic and semantic regularities alleviate difficulties with particularly complex morpho-phonological systems. We further show that our bootstrapping methods substantially outperform hallucination-based methods commonly used for overcoming the annotation bottleneck in morphological reinflection tasks.
