Previous research has found that task-oriented conversational agents are perceived more positively by users when they provide information in an empathetic manner compared to a plain, emotionless information exchange. However, users' perception and ethical considerations related to a dialog systems' response language style have received comparatively little attention in the field of human-computer interaction. To bridge this gap, we explored these ethical implications through a scenario-based user study. 127 participants interacted with one of three variants of an affective, task-oriented conversational agent, each variant providing responses in a different language style. After the interaction, participants filled out a survey about their feelings during the experiment and their perception of various aspects of the chatbot. Based on statistical and qualitative analysis of the responses, we found language style played an important role in how human-like participants perceived a dialog agent as well as how likable. Language style also had a direct effect on how users perceived the use of personal pronouns `I' and `You' and how they projected gender onto the chatbot. Finally, we identify and discuss ethical implications. In particular we focus on what factors/stereotypes influenced participants' impressions of gender, and what trade-offs a more human-like chatbot brings.
