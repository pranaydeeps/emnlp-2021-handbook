Many open-domain question answering problems can be cast as a textual entailment task, where a question and candidate answers are concatenated to form hypotheses. A QA system then determines if the supporting knowledge bases, regarded as potential premises, entail the hypotheses. In this paper, we investigate a neural-symbolic QA approach that integrates natural logic reasoning within deep learning architectures, towards developing effective and yet explainable question answering models. The proposed model gradually bridges a hypothesis and candidate premises following natural logic inference steps to build proof paths. Entailment scores between the acquired intermediate hypotheses and candidate premises are measured to determine if a premise entails the hypothesis. As the natural logic reasoning process forms a tree-like, hierarchical structure, we embed hypotheses and premises in a Hyperbolic space rather than Euclidean space to acquire more precise representations. Empirically, our method outperforms prior work on answering multiple-choice science questions, achieving the best results on two publicly available datasets. The natural logic inference process inherently provides evidence to help explain the prediction process.
