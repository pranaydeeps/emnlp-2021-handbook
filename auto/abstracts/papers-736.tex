Data-driven subword segmentation has become the default strategy for open-vocabulary machine translation and other NLP tasks, but may not be sufficiently generic for optimal learning of non-concatenative morphology. We design a test suite to evaluate segmentation strategies on different types of morphological phenomena in a controlled, semi-synthetic setting. In our experiments, we compare how well machine translation models trained on subword- and character-level can translate these morphological phenomena. We find that learning to analyse and generate morphologically complex surface representations is still challenging, especially for non-concatenative morphological phenomena like reduplication or vowel harmony and for rare word stems. Based on our results, we recommend that novel text representation strategies be tested on a range of typologically diverse languages to minimise the risk of adopting a strategy that inadvertently disadvantages certain languages.
