In the current study, we analyzed 15297 texts from 39 cancer survivors who posted or commented on Reddit in order to detect the language particularities of cancer survivors from online discourse. We performed a computational linguistic analysis (part-of-speech analysis, emoji detection, sentiment analysis) on submissions around the time of the cancer diagnosis and around the time of remission. We found several significant differences in the texts posted around the time of remission compared to those around the time of diagnosis. Though our results need to be backed up by a higher corpus of data, they do cue to the fact that cancer survivors, around the time of remission, focus more on others, are more active on social media, and do not see the glass as half empty as suggested by the valence of the emojis.
