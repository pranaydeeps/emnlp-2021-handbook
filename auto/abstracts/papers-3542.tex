Our goal, in the context of open-domain textual question-answering (QA), is to explain answers by showing the line of reasoning from what is known to the answer, rather than simply showing a fragment of textual evidence (a ``rationale''). If this could be done, new opportunities for understanding and debugging the system's reasoning become possible. Our approach is to generate explanations in the form of entailment trees, namely a tree of multipremise entailment steps from facts that are known, through intermediate conclusions, to the hypothesis of interest (namely the question + answer). To train a model with this skill, we created ENTAILMENTBANK, the first dataset to contain multistep entailment trees. Given a hypothesis (question + answer), we define three increasingly difficult explanation tasks: generate a valid entailment tree given (a) all relevant sentences (b) all relevant and some irrelevant sentences, or (c) a corpus. We show that a strong language model can partially solve these tasks, in particular when the relevant sentences are included in the input (e.g., 35\% of trees for (a) are perfect), and with indications of generalization to other domains. This work is significant as it provides a new type of dataset (multistep entailments) and baselines, offering a new avenue for the community to generate richer, more systematic explanations.
