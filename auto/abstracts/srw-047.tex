Discourse markers (DMs) are ubiquitous cohesive devices used to connect what is said or written. However, across languages there is divergence in their usage, placement, and frequency, which is considered to be a major problem for machine translation (MT). This paper presents an overview of a proposed thesis, exploring the difficulties around DMs in MT, with a focus on Chinese and English. The thesis will examine two main areas: modelling cohesive devices within sentences and modelling discourse relations (DRs) across sentences. Initial experiments have shown promising results for building a prediction model that uses linguistically inspired features to help improve word alignments with respect to the implicit use of cohesive devices, which in turn leads to improved hierarchical phrase based MT.
