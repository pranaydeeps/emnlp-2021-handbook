In multimodal sentiment analysis (MSA), the performance of a model highly depends on the quality of synthesized embeddings. These embeddings are generated from the upstream process called multimodal fusion, which aims to extract and combine the input unimodal raw data to produce a richer multimodal representation. Previous work either back-propagates the task loss or manipulates the geometric property of feature spaces to produce favorable fusion results, which neglects the preservation of critical task-related information that flows from input to the fusion results. In this work, we propose a framework named MultiModal InfoMax (MMIM), which hierarchically maximizes the Mutual Information (MI) in unimodal input pairs (inter-modality) and between multimodal fusion result and unimodal input in order to maintain task-related information through multimodal fusion. The framework is jointly trained with the main task (MSA) to improve the performance of the downstream MSA task. To address the intractable issue of MI bounds, we further formulate a set of computationally simple parametric and non-parametric methods to approximate their truth value. Experimental results on the two widely used datasets demonstrate the efficacy of our approach.
