Most of the time, when dealing with a particular Natural Language Processing task, systems are compared on the basis of global statistics such as recall, precision, F1-score, etc. While such scores provide a general idea of the behavior of these systems, they ignore a key piece of information that can be useful for assessing progress and discerning remaining challenges: the relative difficulty of test instances. To address this shortcoming, we introduce the notion of differential evaluation which effectively defines a pragmatic partition of instances into gradually more difficult bins by leveraging the predictions made by a set of systems. Comparing systems along these difficulty bins enables us to produce a finer-grained analysis of their relative merits, which we illustrate on two use-cases: a comparison of systems participating in a multi-label text classification task (CLEF eHealth 2018 ICD-10 coding), and a comparison of neural models trained for biomedical entity detection (BioCreative V chemical-disease relations dataset).
