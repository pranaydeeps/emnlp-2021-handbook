Any test that promises to assess Human Knowledge of Language (KoL) for any statistically-based Language Model (LM) must meet three requirements: (1) comprehensive coverage of linguistic phenomena; (2) replicable and statistically-vetted human judgement data; and (3) test the LM's ability to track the gradience of sentence acceptability.  To this end, we propose here the LI-Adger dataset: a comprehensive collection of 519 sentence types (4177 sentences) spanning the field of current generative linguistics, accompanied by attested and replicable human acceptability judgements (Sprouse \& Almeida, 2012; Sprouse et al. 2013; Sprouse \& Almeida, 2017).  Finally, we posit the Acceptability Delta Criterion (ADC), an evaluation metric that tests how well a LM can track changes in human acceptability judgements across minimal pairs instead of testing whether the LM assigned a greater likelihood to the expert-labeled acceptable sequence of a minimal pair (S\_1 > S\_2).  We benchmark six different BERT (Devlin et al. 2018) models and a baseline trigram model with the ADC.  Although the best performing BERT model scores 94\%, and the trigram scores 75\% classification accuracy under the traditional metric, performance drops precipitously to 38\% for BERT and 30\% for the trigram model under the ADC.  Adopting the ADC reveals how much harder it is for LMs to track the gradience of acceptability across minimal pairs.  With this work, we propose and provide the three necessary requirements for a comprehensive linguistic analysis and test of the apparently Human KoL exhibited by LMs that we believe is currently missing in the field of Computational Linguistics.
