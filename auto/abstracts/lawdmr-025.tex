While annotating normalized times in food security documents, we found that the semantically compositional annotation for time normalization (SCATE) scheme required several near-duplicate annotations to get the correct semantics for expressions like Nov. 7th to 11th 2021. To reduce this problem, we explored replacing SCATE's Sub-Interval property with a Super-Interval property, that is, making the smallest units (e.g., 7th and 11th) rather than the largest units (e.g., 2021) the heads of the intersection chains. To ensure that the semantics of annotated time intervals remained unaltered despite our changes to the syntax of the annotation scheme, we applied several different techniques to validate our changes. These validation techniques detected and allowed us to resolve several important bugs in our automated translation from Sub-Interval to Super-Interval syntax.
