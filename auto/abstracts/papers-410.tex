Empathy is a complex cognitive ability based on the reasoning of others' affective states. In order to better understand others and express stronger empathy in dialogues, we argue that two issues must be tackled at the same time: (i) identifying which word is the cause for the other's emotion from his or her utterance and (ii) reflecting those specific words in the response generation. However, previous approaches for recognizing emotion cause words in text require sub-utterance level annotations, which can be demanding. Taking inspiration from social cognition, we leverage a generative estimator to infer emotion cause words from utterances with no word-level label. Also, we introduce a novel method based on pragmatics to make dialogue models focus on targeted words in the input during generation. Our method is applicable to any dialogue models with no additional training on the fly. We show our approach improves multiple best-performing dialogue agents on generating more focused empathetic responses in terms of both automatic and human evaluation.
