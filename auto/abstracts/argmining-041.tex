We propose a methodology for representing the reasoning structure of arguments using Bayesian networks and predicate logic facilitated by argumentation schemes. We express the meaning of text segments using predicate logic and map the boolean values of predicate logic expressions to nodes in a Bayesian network. The reasoning structure among text segments is described with a directed acyclic graph. While our formalism is highly expressive and capable of describing the informal logic of human arguments, it is too open-ended to actually build a network for an argument. It is not at all obvious which segment of argumentative text should be considered as a node in a Bayesian network, and how to decide the dependencies among nodes. To alleviate the difficulty, we provide abstract network fragments, called idioms, which represent typical argument justification patterns derived from argumentation schemes. The network construction process is decomposed into idiom selection, idiom instantiation, and idiom combination. We define 17 idioms in total by referring to argumentation schemes as well as analyzing actual arguments and fitting idioms to them. We also create a dataset consisting of pairs of an argumentative text and a corresponding Bayesian network. Our dataset contains about 2,400 pairs, which is large in the research area of argumentation schemes.
