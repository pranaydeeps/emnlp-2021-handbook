Previous works have shown that contextual information can improve the performance of neural machine translation (NMT). However, most existing document-level NMT methods failed to leverage contexts beyond a few set of previous sentences. How to make use of the whole document as global contexts is still a challenge. To address this issue, we hypothesize that a document can be represented as a graph that connects relevant contexts regardless of their distances. We employ several types of relations, including adjacency, syntactic dependency, lexical consistency, and coreference, to construct the document graph. Then, we incorporate both source and target graphs into the conventional Transformer architecture with graph convolutional networks. Experiments on various NMT benchmarks, including IWSLT English--French, Chinese-English, WMT English--German and Opensubtitle English--Russian, demonstrate that using document graphs can significantly improve the translation quality. Extensive analysis verifies that the document graph is beneficial for capturing discourse phenomena.
