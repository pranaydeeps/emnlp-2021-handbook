In interpretable NLP, we require faithful rationales that reflect the model's decision-making process for an explained instance. While prior work focuses  on extractive rationales (a subset of the input words), we investigate their less-studied counterpart: free-text natural language rationales. We demonstrate that *pipelines*, models for faithful rationalization on information-extraction style tasks, do not work as well on ``reasoning'' tasks requiring free-text rationales. We turn to models that *jointly* predict and rationalize, a class of widely used high-performance models for free-text rationalization. We investigate the extent to which the labels and rationales predicted by these models are associated, a necessary property of faithful explanation. Via two tests, *robustness equivalence* and *feature importance agreement*, we find that state-of-the-art T5-based joint models exhibit desirable properties for explaining commonsense question-answering and natural language inference, indicating their potential for producing faithful free-text rationales.
