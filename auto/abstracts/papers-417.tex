Natural language generation (NLG) benchmarks provide an important avenue to measure progress and develop better NLG systems. Unfortunately, the lack of publicly available NLG benchmarks for low-resource languages poses a challenging barrier for building NLG systems that work well for languages with limited amounts of data. Here we introduce IndoNLG, the first benchmark to measure natural language generation (NLG) progress in three low-resource---yet widely spoken---languages of Indonesia: Indonesian, Javanese, and Sundanese. Altogether, these languages are spoken by more than 100 million native speakers, and hence constitute an important use case of NLG systems today. Concretely, IndoNLG covers six tasks: summarization, question answering, chit-chat, and three different pairs of machine translation (MT) tasks. We collate a clean pretraining corpus of Indonesian, Sundanese, and Javanese datasets, Indo4B-Plus, which is used to pretrain our models: IndoBART and IndoGPT. We show that IndoBART and IndoGPT achieve competitive performance on all tasks---despite using only one-fifth the parameters of a larger multilingual model, mBART-large (Liu et al., 2020). This finding emphasizes the importance of pretraining on closely related, localized languages to achieve more efficient learning and faster inference at very low-resource languages like Javanese and Sundanese.
