Deriving and modifying graphs from natural language text has become a versatile basis technology for information extraction with applications in many subfields, such as semantic parsing or knowledge graph construction. A recent work used this technique for modifying scene graphs (He et al. 2020), by first encoding the original graph and then generating the modified one based on this encoding. In this work, we show that we can considerably increase performance on this problem by phrasing it as graph extension instead of graph generation. We propose the first model for the resulting graph extension problem based on autoregressive sequence labelling. On three scene graph modification data sets, this formulation leads to improvements in accuracy over the state-of-the-art between 13 and 24 percentage points. Furthermore, we introduce a novel data set from the biomedical domain which has much larger linguistic variability and more complex graphs than the scene graph modification data sets. For this data set, the state-of-the art fails to generalize, while our model can produce meaningful predictions.
