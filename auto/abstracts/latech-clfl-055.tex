This paper applies stylometry to quantify the literariness of 73 novels and novellas by American author Stephen King, chosen as an extraordinary case of a writer who has been dubbed both ``high'' and ``low'' in literariness in critical reception. We operationalize literariness using a measure of stylistic distance (Cosine Delta) based on the 1000 most frequent words in two bespoke comparison corpora used as proxies for literariness: one of popular genre fiction, another of National Book Award-winning authors. We report that a supervised model is highly effective in distinguishing the two categories, with 94.6\% macro average in a binary classification. We define two subsets of texts by King---``high'' and ``low'' literariness works as suggested by critics and ourselves---and find that a predictive model does identify King's Dark Tower series and novels such as Dolores Claiborne as among his most ``literary'' texts, consistent with critical reception, which has also ascribed postmodern qualities to the Dark Tower novels. Our results demonstrate the efficacy of Cosine Delta-based stylometry in quantifying the literariness of texts, while also highlighting the methodological challenges of literariness, especially in the case of Stephen King. The code and data to reproduce our results are available at https://github.com/andreasvc/kinglit
