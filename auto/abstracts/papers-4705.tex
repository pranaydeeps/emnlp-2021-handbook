Learning sentence embeddings from dialogues has drawn increasing attention due to its low annotation cost and high domain adaptability. Conventional approaches employ the siamese-network for this task, which obtains the sentence embeddings through modeling the context-response semantic relevance by applying a feed-forward network on top of the sentence encoders. However, as the semantic textual similarity is commonly measured through the element-wise distance metrics (e.g. cosine and L2 distance), such architecture yields a large gap between training and evaluating. In this paper, we propose DialogueCSE, a dialogue-based contrastive learning approach to tackle this issue. DialogueCSE first introduces a novel matching-guided embedding (MGE) mechanism, which generates a context-aware embedding for each candidate response embedding (i.e. the context-free embedding) according to the guidance of the multi-turn context-response matching matrices. Then it pairs each context-aware embedding with its corresponding context-free embedding and finally minimizes the contrastive loss across all pairs. We evaluate our model on three multi-turn dialogue datasets: the Microsoft Dialogue Corpus, the Jing Dong Dialogue Corpus, and the E-commerce Dialogue Corpus. Evaluation results show that our approach significantly outperforms the baselines across all three datasets in terms of MAP and Spearman's correlation measures, demonstrating its effectiveness. Further quantitative experiments show that our approach achieves better performance when leveraging more dialogue context and remains robust when less training data is provided.
