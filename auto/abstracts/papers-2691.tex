As a prominent attribution-based explanation algorithm, Integrated Gradients (IG) is widely adopted due to its desirable explanation axioms and the ease of gradient computation. It measures feature importance by averaging the model's output gradient interpolated along a straight-line path in the input data space. However, such straight-line interpolated points are not representative of text data due to the inherent discreteness of the word embedding space. This questions the faithfulness of the gradients computed at the interpolated points and consequently, the quality of the generated explanations. Here we propose Discretized Integrated Gradients (DIG), which allows effective attribution along non-linear interpolation paths. We develop two interpolation strategies for the discrete word embedding space that generates interpolation points that lie close to actual words in the embedding space, yielding more faithful gradient computation. We demonstrate the effectiveness of DIG over IG through experimental and human evaluations on multiple sentiment classification datasets. We provide the source code of DIG to encourage reproducible research.
