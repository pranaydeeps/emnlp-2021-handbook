In this paper, we investigate what types of stereotypical information are captured by pretrained language models. We present the first dataset comprising stereotypical attributes of a range of social groups and propose a method to elicit stereotypes encoded by pretrained language models in an unsupervised fashion. Moreover, we link the emergent stereotypes to their manifestation as basic emotions as a means to study their emotional effects in a more generalized manner. To demonstrate how our methods can be used to analyze emotion and stereotype shifts due to linguistic experience, we use fine-tuning on news sources as a case study. Our experiments expose how attitudes towards different social groups vary across models and how quickly emotions and stereotypes can shift at the fine-tuning stage.
