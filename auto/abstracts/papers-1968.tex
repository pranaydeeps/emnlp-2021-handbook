The task of learning from only a few examples (called a few-shot setting) is of key importance and relevance to a real-world setting. For question answering (QA), the current state-of-the-art pre-trained models typically need fine-tuning on tens of thousands of examples to obtain good results. Their performance degrades significantly in a few-shot setting (< 100 examples). To address this, we propose a simple fine-tuning framework that leverages pre-trained text-to-text models and is directly aligned with their pre-training framework. Specifically, we construct the input as a concatenation of the question, a mask token representing the answer span and a context. Given this input, the model is fine-tuned using the same objective as that of its pre-training objective. Through experimental studies on various few-shot configurations, we show that this formulation leads to significant gains on multiple QA benchmarks (an absolute gain of 34.2 F1 points on average when there are only 16 training examples). The gains extend further when used with larger models (Eg:- 72.3 F1 on SQuAD using BART-large with only 32 examples) and translate well to a multilingual setting . On the multilingual TydiQA benchmark, our model outperforms the XLM-Roberta-large by an absolute margin of upto 40 F1 points and an average of 33 F1 points in a few-shot setting (<= 64 training examples). We conduct detailed ablation studies to analyze factors contributing to these gains.
