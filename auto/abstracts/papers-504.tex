To translate natural language questions into executable database queries, most approaches rely on a fully annotated training set. Annotating a large dataset with queries is difficult as it requires query-language expertise. We reduce this burden using grounded in databases intermediate question representations. These representations are simpler to collect and were originally crowdsourced within the Break dataset (Wolfson et al., 2020). Our pipeline consists of two parts: a neural semantic parser that converts natural language questions into the intermediate representations and a non-trainable transpiler to the SPARQL query language (a standard language for accessing knowledge graphs and semantic web). We chose SPARQL because its queries are structurally closer to our intermediate representations (compared to SQL). We observe that the execution accuracy of queries constructed by our model on the challenging Spider dataset is comparable with the state-of-the-art text-to-SQL methods trained with annotated SQL queries. Our code and data are publicly available (https://github.com/yandex-research/sparqling-queries).
