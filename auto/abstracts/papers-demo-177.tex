When journalists cover a news story, they can cover the story from multiple angles or perspectives. These perspectives are called ``frames'' and usage of one frame or another may influence public perception and opinion of the issue at hand. We develop a web-based system for analyzing frames in multilingual text documents. We propose and guide users through a five-step end-to-end computational framing analysis framework grounded in media framing theory in communication research. Users can use the framework to analyze multilingual text data, starting from the exploration of frames in user corpora and through review of previous framing literature (step 1-3) to frame classification (step 4) and prediction (step 5). The framework combines unsupervised and supervised machine learning and leverages a state-of-the-art (SoTA) multilingual language model, which can significantly enhance frame prediction performance while requiring a considerably small sample of manual annotations. Through the interactive website, anyone can perform the proposed computational framing analysis, making advanced computational analysis available to researchers without a programming background and bridging the digital divide within the communication research discipline in particular and the academic community in general. The system is available online at http://www.openframing.org, via an API http://www.openframing.org:5000/docs/, or through our GitHub page https://github.com/vibss2397/openFraming.