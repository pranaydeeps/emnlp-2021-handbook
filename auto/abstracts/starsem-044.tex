Hypernymy relation acquisition has been widely investigated, especially because taxonomies, which often constitute the backbone structure of semantic resources are structured using this type of relations. Although lots of approaches have been dedicated to this task, most of them analyze only the written text. However relations between not necessarily contiguous textual units can be expressed, thanks to typographical or dispositional markers. Such relations, which are out of reach of standard NLP tools, have been investigated in well specified layout contexts. Our aim is to improve the relation extraction task considering both the plain text and the layout. We are proposing here a method which combines layout, discourse and terminological analyses, and performs a structured prediction. We focused on textual structures which correspond to a well defined discourse structure and which often bear hypernymy relations. This type of structure encompasses titles and sub-titles, or enumerative structures. The results achieve a precision of about 60\%.
