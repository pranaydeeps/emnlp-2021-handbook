Common grounding is the process of creating and maintaining mutual understandings, which is a critical aspect of sophisticated human communication. While various task settings have been proposed in existing literature, they mostly focus on creating common ground under static context and ignore the aspect of maintaining them overtime under dynamic context. In this work, we propose a novel task setting to study the ability of both creating and maintaining common ground in dynamic environments. Based on our minimal task formulation, we collected a large-scale dataset of 5,617 dialogues to enable fine-grained evaluation and analysis of various dialogue systems. Through our dataset analyses, we highlight novel challenges introduced in our setting, such as the usage of complex spatio-temporal expressions to create and maintain common ground. Finally, we conduct extensive experiments to assess the capabilities of our baseline dialogue system and discuss future prospects of our research.