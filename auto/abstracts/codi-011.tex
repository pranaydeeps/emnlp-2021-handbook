In data-driven natural language generation, we typically know what relation should be expressed and need to select a connective to lexicalize it. In the current contribution, we analyse whether a sophisticated connective generation module is necessary to select a connective, or whether this can be solved with simple methods (such as random choice between connectives that are known to express a given relation, or usage of a generic language model). Comparing these methods to the distributions of connective choices from a human connective insertion task, we find mixed results: for some relations, it is acceptable to lexicalize them using any of the connectives that mark this relation. However, for other relations (temporals, concessives) either a more detailed relation distinction needs to be introduced, or a more sophisticated connective choice module would be necessary.
