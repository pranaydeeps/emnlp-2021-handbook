Scientific literature analysis needs fine-grained named entity recognition (NER) to provide a wide range of information for scientific discovery. For example, chemistry research needs to study dozens to hundreds of distinct, fine-grained entity types, making consistent and accurate annotation difficult even for crowds of domain experts. On the other hand, domain-specific ontologies and knowledge bases (KBs) can be easily accessed, constructed, or integrated, which makes distant supervision realistic for fine-grained chemistry NER. In distant supervision, training labels are generated by matching mentions in a document with the concepts in the knowledge bases (KBs). However, this kind of KB-matching suffers from two major challenges: incomplete annotation and noisy annotation. We propose ChemNER, an ontology-guided, distantly-supervised method for fine-grained chemistry NER to tackle these challenges. It leverages the chemistry type ontology structure to generate distant labels with novel methods of flexible KB-matching and ontology-guided multi-type disambiguation. It significantly improves the distant label generation for the subsequent sequence labeling model training. We also provide an expert-labeled, chemistry NER dataset with 62 fine-grained chemistry types (e.g., chemical compounds and chemical reactions). Experimental results show that ChemNER is highly effective, outperforming substantially the state-of-the-art NER methods (with .25 absolute F1 score improvement).
