Point-of-interest (POI) type prediction is the task of inferring the type of a place from where a social media post was shared. Inferring a POI's type is useful for studies in computational social science including sociolinguistics, geosemiotics, and cultural geography, and has applications in geosocial networking technologies such as recommendation and visualization systems. Prior efforts in POI type prediction focus solely on text, without taking visual information into account. However in reality, the variety of modalities, as well as their semiotic relationships with one another, shape communication and interactions in social media. This paper presents a study on POI type prediction using multimodal information from text and images available at posting time. For that purpose, we enrich a currently available data set for POI type prediction with the images that accompany the text messages. Our proposed method extracts relevant information from each modality to effectively capture interactions between text and image achieving a macro F1 of 47.21 across 8 categories significantly outperforming the state-of-the-art method for POI type prediction based on text-only methods. Finally, we provide a detailed analysis to shed light on cross-modal interactions and the limitations of our best performing model.
