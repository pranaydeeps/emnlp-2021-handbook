This paper describes the first SemEval task to explore the use of Natural Language Processing systems for building dictionary entries, within the framework of Corpus Pattern Analysis. CPA is a corpus-driven technique which provides tools and resources to identify and represent unambiguously the main semantic patterns in which words are used. This task draws on the Pattern Dictionary of English Verbs (http://www.pdev.org.uk), for the targeted lexical entries, and on the British National Corpus for the input text. Dictionary entry building is split into three subtasks which all start from a concordance sample: 1) CPA parsing, where arguments and their syntactic and semantic categories have to be identified, 2) CPA clustering, in which sentences with similar patterns have to be clustered and 3) CPA automatic lexicography where the structure of patterns have to be constructed automatically. Subtask 1 attracted 3 teams, though none could beat the baseline (rule-based system). Subtask 2 attracted 2 teams, one of which beat the baseline (majority-class classifier).  Subtask 3 did not attract any participant. The task has produced a major semantic multi-dataset resource which includes data for 121 verbs and about 17,000 annotated sentences, and which is freely accessible.
