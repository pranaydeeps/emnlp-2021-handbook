Existing text-based personality detection research mostly relies on data-driven approaches to implicitly capture personality cues in online posts, lacking the guidance of psychological knowledge. Psychological questionnaire, which contains a series of dedicated questions highly related to personality traits, plays a critical role in self-report personality assessment. We argue that the posts created by a user contain critical contents that could help answer the questions in a questionnaire, resulting in an assessment of his personality by linking the texts and the questionnaire. To this end, we propose a new model named Psychological Questionnaire enhanced Network (PQ-Net) to guide personality detection by tracking critical information in texts with a questionnaire. Specifically, PQ-Net contains two streams: a context stream to encode each piece of text into a contextual text representation, and a questionnaire stream to capture relevant information in the contextual text representation to generate potential answer representations for a questionnaire. The potential answer representations are used to enhance the contextual text representation and to benefit personality prediction. Experimental results on two datasets demonstrate the superiority of PQ-Net in capturing useful cues from the posts for personality detection.
