Neural language models typically tokenise input text into sub-word units to achieve an open vocabulary. The standard approach is to use a single canonical tokenisation at both train and test time. We suggest that this approach is unsatisfactory and may bottleneck our evaluation of language model performance. Using only the one-best tokenisation ignores tokeniser uncertainty over alternative tokenisations, which may hurt model out-of-domain performance. In this paper, we argue that instead, language models should be evaluated on their marginal likelihood over tokenisations. We compare different estimators for the marginal likelihood based on sampling, and show that it is feasible to estimate the marginal likelihood with a manageable number of samples. We then evaluate a pretrained language model on both the one-best-tokenisation and marginal perplexities, and show that the marginal perplexity can be significantly better than the one best, especially on out-of-domain data. We link this difference in perplexity to the tokeniser uncertainty as measured by tokeniser entropy. We discuss some implications of our results for language model training and evaluation, particularly with regard to tokenisation robustness.
