Natural language models often fall short when understanding and generating mathematical notation. What is not clear is whether these shortcomings are due to fundamental limitations of the models, or the absence of appropriate tasks. In this paper, we explore the extent to which natural language models can learn semantics between mathematical notation and their surrounding text. We propose two notation prediction tasks, and train a  model that selectively masks notation tokens and encodes left and/or right sentences as context. Compared to baseline models trained by masked language modeling, our method achieved significantly better performance at the two tasks, showing that this approach is a good first step towards modeling mathematical texts. However, the current models rarely predict unseen symbols correctly, and token-level predictions are more accurate than symbol-level predictions, indicating more work is needed to represent structural patterns. Based on the results, we suggest future works toward modeling mathematical texts.
