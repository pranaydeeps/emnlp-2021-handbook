A ``bigger is better'' explosion in the number of parameters in deep neural networks has made it increasingly challenging to make state-of-the-art networks accessible in compute-restricted environments. Compression techniques have taken on renewed importance as a way to bridge the gap. However, evaluation of the trade-offs incurred by popular compression techniques has been centered on high-resource datasets. In this work, we instead consider the impact of compression in a data-limited regime. We introduce the term low-resource double bind to refer to the co-occurrence of data limitations and compute resource constraints. This is a common setting for NLP for low-resource languages, yet the trade-offs in performance are poorly studied. Our work offers surprising insights into the relationship between capacity and generalization in data-limited regimes for the task of machine translation. Our experiments on magnitude pruning for translations from English into Yoruba, Hausa, Igbo and German show that in low-resource regimes, sparsity preserves performance on frequent sentences but has a disparate impact on infrequent ones. However, it improves robustness to out-of-distribution shifts, especially for datasets that are very distinct from the training distribution. Our findings suggest that sparsity can play a beneficial role at curbing memorization of low frequency attributes, and therefore offers a promising solution to the low-resource double bind.
