Dense retrieval methods have shown great promise over sparse retrieval methods in a range of NLP problems. Among them, dense phrase retrieval—the most fine-grained retrieval unit—is appealing because phrases can be directly used as the output for question answering and slot filling tasks. In this work, we follow the intuition that retrieving phrases naturally entails retrieving larger text blocks and study whether phrase retrieval can serve as the basis for coarse-level retrieval including passages and documents. We first observe that a dense phrase-retrieval system, without any retraining, already achieves better passage retrieval accuracy (+3-5\% in top-5 accuracy) compared to passage retrievers, which also helps achieve superior end-to-end QA performance with fewer passages. Then, we provide an interpretation for why phrase-level supervision helps learn better fine-grained entailment compared to passage-level supervision, and also show that phrase retrieval can be improved to achieve competitive performance in document-retrieval tasks such as entity linking and knowledge-grounded dialogue. Finally, we demonstrate how phrase filtering and vector quantization can reduce the size of our index by 4-10x, making dense phrase retrieval a practical and versatile solution in multi-granularity retrieval.
