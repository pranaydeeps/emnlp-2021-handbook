Despite showing increasingly human-like conversational abilities, state-of-the-art dialogue models often suffer from factual incorrectness and hallucination of knowledge (Roller et al., 2020). In this work we explore the use of neural-retrieval-in-the-loop architectures - recently shown to be effective in open-domain QA (Lewis et al., 2020b; Izacard and Grave, 2020) - for knowledge-grounded dialogue, a task that is arguably more challenging as it requires querying based on complex multi-turn dialogue context and generating conversationally coherent responses. We study various types of architectures with multiple components - retrievers, rankers, and encoder-decoders - with the goal of maximizing knowledgeability while retaining conversational ability. We demonstrate that our best models obtain state-of-the-art performance on two knowledge-grounded conversational tasks. The models exhibit open-domain conversational capabilities, generalize effectively to scenarios not within the training data, and, as verified by human evaluations, substantially reduce the well-known problem of knowledge hallucination in state-of-the-art chatbots.
