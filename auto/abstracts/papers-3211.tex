Stance detection, which aims to determine whether an individual is for or against a target concept, promises to uncover public opinion from large streams of social media data. Yet even human annotation of social media content does not always capture ``stance'' as measured by public opinion polls. We demonstrate this by directly comparing an individual's self-reported stance to the stance inferred from their social media data. Leveraging a longitudinal public opinion survey with respondent Twitter handles, we conducted this comparison for 1,129 individuals across four salient targets. We find that recall is high for both ``Pro'' and ``Anti'' stance classifications but precision is variable in a number of cases. We identify three factors leading to the disconnect between text and author stance: temporal inconsistencies, differences in constructs, and measurement errors from both survey respondents and annotators. By presenting a framework for assessing the limitations of stance detection  models, this work provides important insight into what stance detection truly measures.
