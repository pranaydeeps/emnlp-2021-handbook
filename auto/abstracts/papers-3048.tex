Psychometric measures of ability, attitudes, perceptions, and beliefs are crucial for understanding user behavior in various contexts including health, security, e-commerce, and finance. Traditionally, psychometric dimensions have been measured and collected using survey-based methods. Inferring such constructs from user-generated text could allow timely, unobtrusive collection and analysis. In this paper we describe our efforts to construct a corpus for psychometric natural language processing (NLP) related to important dimensions such as trust, anxiety, numeracy, and literacy, in the health domain. We discuss our multi-step process to align user text with their survey-based response items and provide an overview of the resulting testbed which encompasses survey-based psychometric measures and accompanying user-generated text from 8,502 respondents. Our testbed also encompasses self-reported demographic information, including race, sex, age, income, and education - thereby affording opportunities for measuring bias and benchmarking fairness of text classification methods. We report preliminary results on use of the text to predict/categorize users' survey response labels - and on the fairness of these models. We also discuss the important implications of our work and resulting testbed for future NLP research on psychometrics and fairness.
