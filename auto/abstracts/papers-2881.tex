Coreference resolution is key to many natural language processing tasks and yet has  been relatively unexplored in Sign Language Processing. In signed languages, space is primarily used to establish reference. Solving coreference resolution for signed languages would not only enable higher-level Sign Language Processing systems, but also enhance our understanding of language in different modalities and of situated references, which are key problems in studying grounded language. In this paper, we: (1) introduce Signed Coreference Resolution (SCR), a new challenge for coreference modeling and Sign Language Processing; (2) collect an annotated corpus of German Sign Language with gold labels for coreference together with an annotation software for the task; (3) explore features of hand gesture, iconicity, and spatial situated properties and move forward to propose a set of linguistically informed heuristics and unsupervised models for the task; (4) put forward several proposals about ways to address the complexities of this challenge effectively.
