In this paper, we propose to use active learning for training classifiers to judge the quality of gap-fill questions. Gap-fill questions are widely used for assessments in education contexts because they can be graded automatically while offering reliable assessment of learners' knowledge level if appropriately calibrated. Active learning is a machine learning framework which is typically used when unlabeled data is abundant but annotation is slow and expensive. This is the case in many Natural Language Processing tasks, including automated question generation, which is our focus. A key task in automated question generation is judging the quality of the generated questions. Classifiers can be build to address this tasks which typically are trained on human labeled data. Our evaluation results suggest that the use of active learning leads to accurate classifiers for judging the quality of gap-fill questions while keeping the annotation costs in check. We are not aware of any previous effort that uses active learning for question generation.
