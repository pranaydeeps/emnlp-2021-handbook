Rumor detection on social media puts pre-trained language models (LMs), such as BERT, and auxiliary features, such as comments, into use. However, on the one hand, rumor detection datasets in Chinese companies with comments are rare; on the other hand, intensive interaction of attention on Transformer-based models like BERT may hinder performance improvement. To alleviate these problems, we build a new Chinese microblog dataset named Weibo20 by collecting posts and associated comments from Sina Weibo and propose a new ensemble named STANKER (Stacking neTwork bAsed-on atteNtion-masKed BERT). STANKER adopts two level-grained attention-masked BERT (LGAM-BERT) models as base encoders. Unlike the original BERT, our new LGAM-BERT model takes comments as important auxiliary features and masks co-attention between posts and comments on lower-layers. Experiments on Weibo20 and three existing social media datasets showed that STANKER  outperformed all compared models, especially beating the old state-of-the-art on Weibo dataset.
