Several computational linguistics techniques are applied to analyze a large corpus of Spanish sonnets from the 16th and 17th centuries. The analysis is focused on metrical and semantic aspects. First, we are developing a hybrid scansion system in order to extract and analyze rhythmical or metrical patterns. The possible metrical patterns of each verse are extracted with language-based rules. Then statistical rules are used to resolve ambiguities. Second, we are applying distributional semantic models in order to, on one hand, extract semantic regularities from sonnets, and on the other hand to group together sonnets and poets according to these semantic regularities. Besides these techniques, in this position paper we will show the objectives of the project and partial results.
