Graph-to-text generation aims to generate fluent texts from graph-based data. In this paper, we investigate two recent pretrained language models (PLMs) and analyze the impact of different task-adaptive pretraining strategies for PLMs in graph-to-text generation. We present a study across three graph domains: meaning representations, Wikipedia knowledge graphs (KGs) and scientific KGs. We show that approaches based on PLMs BART and T5 achieve new state-of-the-art results and that task-adaptive pretraining strategies improve their performance even further. We report new state-of-the-art BLEU scores of 49.72 on AMR-LDC2017T10, 59.70 on WebNLG, and 25.66 on AGENDA datasets - a relative improvement of 31.8\%, 4.5\%, and 42.4\%, respectively, with our models generating significantly more fluent texts than human references.  In an extensive analysis, we identify possible reasons for the PLMs' success on graph-to-text tasks. Our findings suggest that the PLMs benefit from similar facts seen during pretraining or fine-tuning, such that they perform well even when the input graph is reduced to a simple bag of node and edge labels.
