Tracing the influence of individuals or groups in social networks is an increasingly popular task in sociolinguistic studies. While methods to determine someone's influence in shortterm contexts (e.g., social media, on-line political debates) are widespread, influence in longterm contexts is less investigated and may be harder to capture. We study the diffusion of scientific terms in an English diachronic scientific corpus, applying Hawkes Processes to capture the role of individual scientists as ``influencers'' or ``influencees'' in the diffusion of new concepts. Our findings on two major scientific discoveries in chemistry and astronomy of the 18th century reveal that modelling both the introduction and diffusion of scientific terms in a historical corpus as Hawkes Processes allows detecting patterns of influence between authors on a long-term scale.
