We explore Few-Shot Learning (FSL) for Relation Classification (RC). Focusing on the realistic scenario of FSL, in which a test instance might not belong to any of the target categories (none-of-the-above, aka NOTA), we first revisit the recent popular dataset structure for FSL, pointing out its unrealistic data distribution. To remedy this, we propose a novel methodology for deriving more realistic few-shot test data from available datasets for supervised RC, and apply it to the TACRED dataset. This yields a new challenging benchmark for FSL RC, on which state of the art models show poor performance. Next, we analyze classification schemes within the popular embedding-based nearest-neighbor approach for FSL, with respect to constraints they impose on the embedding space. Triggered by this analysis we propose a novel classification scheme, in which the NOTA category is represented as learned vectors, shown empirically to be an appealing option for FSL.