This thesis explores the computational structure of morphological paradigms from the perspective of unsupervised learning. Three topics are studied: (i) stem identification, (ii) paradigmatic similarity, and (iii) paradigm induction. All the three topics progress in terms of the scope of data in question. The first and second topics explore structure when morphological paradigms are given, first within a paradigm and then across paradigms. The third topic asks where morphological paradigms come from in the first place, and explores strategies of paradigm induction from child-directed speech. This research is of interest to linguists and natural language processing researchers, for both theoretical questions and applied areas.
