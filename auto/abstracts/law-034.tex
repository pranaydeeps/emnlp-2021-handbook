Return-on-Investment (ROI) is a cost-conscious approach to active learning (AL) that considers both estimates of cost and of benefit in active sample selection. We investigate the theoretical conditions for successful cost-conscious AL using ROI by examining the conditions under which ROI would optimize the area under the cost/benefit curve. We then empirically measure the degree to which optimality is jeopardized in practice when the conditions are violated. The reported experiments involve an English part-of-speech annotation task. Our results show that ROI can indeed successfully reduce total annotation costs and should be considered as a viable option for machine-assisted annotation. On the basis of our experiments, we make recommendations for benefit estimators to be employed in ROI. In particular, we find that the more linearly related a benefit estimate is to the true benefit, the better the estimate performs when paired in ROI with an imperfect cost estimate. Lastly, we apply our analysis to help explain the mixed results of previous work on these questions.
