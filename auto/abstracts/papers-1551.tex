This paper studies the keyphrase generation (KG) task for scenarios where structure plays an important role. For example, a scientific publication consists of a short title and a long body, where the title can be used for de-emphasizing unimportant details in the body. Similarly, for short social media posts (\eg, tweets), scarce context can be augmented from titles, though often missing. Our contribution is generating/augmenting structure then injecting these information in the encoding, using existing keyphrases of other documents, complementing missing/incomplete titles. We propose novel structure-augmented document encoding approaches that consist of the following two phases: The first phase, generating structure, extends the given document with related but absent keyphrases, augmenting missing context. The second phase, encoding structure, builds a graph of keyphrases and the given document to obtain the structure-aware representation of the augmented text. Our empirical results validate that our proposed structure augmentation and augmentation-aware encoding/decoding can improve KG for both scenarios, outperforming the state-of-the-art.
