We propose a new approach for learning contextualised cross-lingual word embeddings based on a small parallel corpus (e.g. a few hundred sentence pairs). Our method obtains word embeddings via an LSTM encoder-decoder model that simultaneously translates and reconstructs an input sentence. Through sharing model parameters among different languages, our model jointly trains the word embeddings in a common cross-lingual space. We also propose to combine word and subword embeddings to make use of orthographic similarities across different languages. We base our experiments on real-world data from endangered languages, namely Yongning Na, Shipibo-Konibo, and Griko. Our experiments on bilingual lexicon induction and word alignment tasks show that our model outperforms existing methods by a large margin for most language pairs. These results demonstrate that, contrary to common belief, an encoder-decoder translation model is beneficial for learning cross-lingual representations even in extremely low-resource conditions. Furthermore, our model also works well on high-resource conditions, achieving state-of-the-art performance on a German-English word-alignment task.
