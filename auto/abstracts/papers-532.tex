We present our work on leveraging multilingual parallel corpora of small sizes for Statistical Machine Translation between Japanese and Hindi using multiple pivot languages. In our setting, the source and target part of the corpus remains the same, but we show that using several different pivot to extract phrase pairs from these source and target parts lead to large BLEU improvements. We focus on a variety of ways to exploit phrase tables generated using multiple pivots to support a direct source-target phrase table. Our main method uses the Multiple Decoding Paths (MDP) feature of Moses, which we empirically verify as the best compared to the other methods we used. We compare and contrast our various results to show that one can overcome the limitations of small corpora by using as many pivot languages as possible in a multilingual setting. Most importantly, we show that such pivoting aids in learning of additional phrase pairs which are not learned when the direct source-target corpus is small. We obtained improvements of up to 3 BLEU points using multiple pivots for Japanese to Hindi translation compared to when only one pivot is used. To the best of our knowledge, this work is also the first of its kind to attempt the simultaneous utilization of 7 pivot languages at decoding time.
