In practical applications of semantic parsing, we often want to rapidly change the behavior of the parser, such as enabling it to handle queries in a new domain, or changing its predictions on certain targeted queries. While we can introduce new training examples exhibiting the target behavior, a mechanism for enacting such behavior changes without expensive model re-training would be preferable. To this end, we propose ControllAble Semantic Parser via Exemplar Retrieval (CASPER). Given an input query, the parser retrieves related exemplars from a retrieval index, augments them to the query, and then applies a generative seq2seq model to produce an output parse. The exemplars act as a control mechanism over the generic generative model: by manipulating the retrieval index or how the augmented query is constructed, we can manipulate the behavior of the parser. On the MTOP dataset, in addition to achieving state-of-the-art on the standard setup, we show that CASPER can parse queries in a new domain, adapt the prediction toward the specified patterns, or adapt to new semantic schemas without having to further re-train the model.
