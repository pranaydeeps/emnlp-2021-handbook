The recent explosion of social media services like Twitter, Google+ and Facebook has led to an in- terest in social media predictive analytics --- automatically inferring hidden information from the large amounts of freely available content. It has a number of applications, including: online targeted advertis- ing, personalized marketing, large-scale passive polling and real-time live polling, personalized recom- mendation systems and search, and real-time healthcare analytics etc. In this tutorial, we will describe how to build a variety of social media predictive analytics for in- ferring latent user properties from a Twitter network including demographic traits, personality, interests, emotions and opinions etc. Our methods will address several important aspects of social media such as: dynamic, streaming nature of the data, multi-relationality in social networks, data collection and anno- tation biases, data and model sharing, generalization of the existing models, data drift, and scalability to other languages. We will start with an overview of the existing approaches for social media predictive analytics. We will describe the state-of-the-art static (batch) models and features. We will then present models for streaming (online) inference from single and multiple data streams; and formulate a latent attribute pre- diction task as a sequence-labeling problem. Finally, we present several techniques for dynamic (iterative) learning and prediction using active learning setup with rationale annotation and filtering. The tutorial will conclude with a practice session focusing on walk-through examples for predicting latent user properties e.g., political preferences, income, education level, life satisfaction and emotions emanating from user communications on Twitter.
