Mathematical reasoning aims to infer satisfiable solutions based on the given mathematics questions. Previous natural language processing researches have proven the effectiveness of sequence-to-sequence (Seq2Seq) or related variants on mathematics solving. However, few works have been able to explore structural or syntactic information hidden in expressions (e.g., precedence and associativity). This dissertation set out to investigate the usefulness of such untapped information for neural architectures. Firstly, mathematical questions are represented in the format of graphs within syntax analysis. The structured nature of graphs allows them to represent relations of variables or operators while preserving the semantics of the expressions. Having transformed to the new representations, we proposed a graph-to-sequence neural network GraphMR, which can effectively learn the hierarchical information of graphs inputs to solve mathematics and speculate answers. A complete experimental scenario with four classes of mathematical tasks and three Seq2Seq baselines is built to conduct a comprehensive analysis, and results show that GraphMR outperforms others in hidden information learning and mathematics resolving.
