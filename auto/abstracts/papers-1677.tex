Biases continue to be prevalent in modern text and media, especially subjective bias --- a special type of bias that introduces improper attitudes or presents a statement with the presupposition of truth. To tackle the problem of detecting and further mitigating subjective bias, we introduce a manually annotated parallel corpus WIKIBIAS with more than 4,000 sentence pairs from Wikipedia edits. This corpus contains annotations towards both sentence-level bias types and token-level biased segments. We present systematic analyses of our dataset and results achieved by a set of state-of-the-art baselines in terms of three tasks: bias classification, tagging biased segments, and neutralizing biased text. We find that current models still struggle with detecting multi-span biases despite their reasonable performances, suggesting that our dataset can serve as a useful research benchmark. We also demonstrate that models trained on our dataset can generalize well to multiple domains such as news and political speeches.
