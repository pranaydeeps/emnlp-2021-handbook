In computational linguistics, it has been shown that hierarchical structures make language models (LMs) more human-like. However, the previous literature has been agnostic about a parsing strategy of the hierarchical models. In this paper, we investigated whether hierarchical structures make LMs more human-like, and if so, which parsing strategy is most cognitively plausible. In order to address this question, we evaluated three LMs against human reading times in Japanese with head-final left-branching structures: Long Short-Term Memory (LSTM) as a sequential model and Recurrent Neural Network Grammars (RNNGs) with top-down and left-corner parsing strategies as hierarchical models. Our computational modeling demonstrated that left-corner RNNGs outperformed top-down RNNGs and LSTM, suggesting that hierarchical and left-corner architectures are more cognitively plausible than top-down or sequential architectures. In addition, the relationships between the cognitive plausibility and (i) perplexity, (ii) parsing, and (iii) beam size will also be discussed.
