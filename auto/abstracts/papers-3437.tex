Recent work has raised concerns about the inherent limitations of text-only pretraining. In this paper, we first demonstrate that reporting bias, the tendency of people to not state the obvious, is one of the causes of this limitation, and then investigate to what extent multimodal training can mitigate this issue. To accomplish this, we 1) generate the Color Dataset (CoDa), a dataset of human-perceived color distributions for 521 common objects; 2) use CoDa to analyze and compare the color distribution found in text, the distribution captured by language models, and a human's perception of color; and 3) investigate the performance differences between text-only and multimodal models on CoDa. Our results show that the distribution of colors that a language model recovers correlates more strongly with the inaccurate distribution found in text than with the ground-truth, supporting the claim that reporting bias negatively impacts and inherently limits text-only training. We then demonstrate that multimodal models can leverage their visual training to mitigate these effects, providing a promising avenue for future research.
