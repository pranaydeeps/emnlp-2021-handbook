Linguistic alignment, such as lexical and syntactic alignment, is a universal phenomenon influencing dialogue participants in online conversations. Besides linguistic alignment, social support, such as emotional support and informational support, also provides important benefits to their members. More recently, identifying and analyzing social support facilitated insights in the construction of support oriented communities. In this paper, we study whether members show pragmatic alignment on social support. While adaptation can occur at lexical, syntactic and pragmatic levels, relationships between alignments at multiple levels are neither theoretically nor empirically well understood. Our results indicate pragmatic alignment of forum members along the axis of support type. We also find that lexical alignment is correlated with emotional support, and that the amount of lexical alignment is also correlated with the amount of pragmatic alignment.  This finding can contribute to improving our understanding about the linguistic signature of different types of support, and enhancing theoretical model of the Interactive Alignment Model in a multi-party peer support conversation context.
