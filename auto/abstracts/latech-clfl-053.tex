We apply statistical techniques from natural language processing to Western and Hong Kong---based English language newspaper articles that discuss the 2019---2020 Hong Kong protests of the Anti-Extradition Law Amendment Bill Movement. Topic modeling detects central themes of the reporting and shows the differing agendas toward \emph{one country, two systems}. Embedding-based usage shift (at the word level) and sentiment analysis (at the document level) both support that Hong Kong---based reporting is more negative and more emotionally charged. A two-way test shows that while July 1, 2019 is a turning point for media portrayal, the differences between western- and Hong Kong---based reporting did not magnify when the protests began; rather, they already existed. Taken together, these findings clarify how the portrayal of activism in Hong Kong evolved throughout the Movement.
