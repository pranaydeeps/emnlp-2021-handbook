To find a suitable embedding for a knowledge graph remains a big challenge nowadays. By using previous knowledge graph embedding methods, every entity in a knowledge graph is usually represented as a k-dimensional vector. As we know, an affine transformation can be expressed in the form of a matrix multiplication followed by a translation vector. In this paper, we firstly utilize a set of affine transformations related to each relation to operate on entity vectors, and then these transformed vectors are used for performing embedding with previous methods. The main advantage of using affine transformations is their good geometry properties with interpretability. Our experimental results demonstrate that the proposed intuitive design with affine transformations provides a statistically significant increase in performance with adding a few extra processing steps or adding a limited number of additional variables. Taking TransE as an example, we employ the scale transformation (the special case of an affine transformation), and only introduce k additional variables for each relation. Surprisingly, it even outperforms RotatE to some extent on various data sets. We also introduce affine transformations into RotatE, Distmult and ComplEx, respectively, and each one outperforms its original method.
