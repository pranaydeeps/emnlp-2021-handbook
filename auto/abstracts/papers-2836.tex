We present a method for exploring regions around individual points in a contextualized vector space (particularly, BERT space), as a way to investigate how these regions correspond to word senses. By inducing a contextualized ``pseudoword'' vector as a stand-in for a static embedding in the input layer, and then performing masked prediction of a word in the sentence, we are able to investigate the geometry of the BERT-space in a controlled manner around individual instances. Using our method on a set of carefully constructed sentences targeting highly ambiguous English words, we find substantial regularity in the contextualized space, with regions that correspond to distinct word senses; but between these regions there are occasionally ``sense voids''—regions that do not correspond to any intelligible sense.
