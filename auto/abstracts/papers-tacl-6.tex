Consistency of a model --- that is, the invariance of its behavior under meaning-preserving alternations in its input --- is a highly desirable property in natural language processing. In this paper we study the question: Are Pretrained Language Models (PLMs) consistent with respect to factual knowledge? To this end, we create ParaRel, a high-quality resource of cloze-style query English paraphrases. It contains a total of 328 paraphrases for 38 relations. Using ParaRel, we show that the consistency of all PLMs we experiment with is poor -- though with high variance between relations. Our analysis of the representational spaces of PLMs suggests that they have a poor structure and are currently not suitable for representing knowledge robustly. Finally, we propose a method for improving model consistency and experimentally demonstrate its effectiveness.