Content-planning is an essential part of data-to-text generation to determine the order of data mentioned in generated texts. Recent neural data-to-text generation models employ Pointer Networks to explicitly learn content-plan given a set of attributes as input. They use LSTM to encode the input, which assumes a sequential relationship in the input. This may be sub-optimal to encode a set of attributes, where the attributes have a composite structure: the attributes are disordered while each attribute value is an ordered list of tokens. We handle this problem by proposing a neural content-planner that can capture both local and global contexts of such a structure. Specifically, we propose a novel attention mechanism called GSC-attention. A key component of the GSC-attention is grouped-attention, which is token-level attention constrained within each input attribute that enables our proposed model captures both local and global context. Moreover, our content-planner explicitly learns content-selection, which is integrated into the content-planner to select the most important data to be included in the generated text via an attention masking procedure. Experimental results show that our model outperforms the competitors by 4.92\%, 4.70\%, and 16.56\% in terms of Damerau-Levenshtein Distance scores on three real-world datasets.
