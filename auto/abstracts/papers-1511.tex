Byte-pair encoding (BPE) is a ubiquitous algorithm in the subword tokenization process of language models as it provides multiple benefits. However, this process is solely based on pre-training data statistics, making it hard for the tokenizer to handle infrequent spellings. On the other hand, though robust to misspellings, pure character-level models often lead to unreasonably long sequences and make it harder for the model to learn meaningful words. To alleviate these challenges, we propose a character-based subword module (char2subword) that learns the subword embedding table in pre-trained models like BERT. Our char2subword module builds representations from characters out of the subword vocabulary, and it can be used as a drop-in replacement of the subword embedding table. The module is robust to character-level alterations such as misspellings, word inflection, casing, and punctuation. We integrate it further with BERT through pre-training while keeping BERT transformer parameters fixed--and thus, providing a practical method. Finally, we show that incorporating our module to mBERT significantly improves the performance on the social media linguistic code-switching evaluation (LinCE) benchmark.
