Many applications require generation of summaries tailored to the user's information needs, i.e., their intent. Methods that express intent via explicit user queries fall short when query interpretation is subjective. Several datasets exist for summarization with objective intents where, for each document and intent (e.g., ``weather''), a single summary suffices for all users. No datasets exist, however, for subjective intents (e.g., ``interesting places'') where different users will provide different summaries. We present SUBSUME, the first dataset for evaluation of SUBjective SUMmary Extraction systems. SUBSUME contains 2,200 (document, intent, summary) triplets over 48 Wikipedia pages, with ten intents of varying subjectivity, provided by 103 individuals over Mechanical Turk. We demonstrate statistically that the intents in SUBSUME vary systematically in subjectivity. To indicate SUBSUME's usefulness, we explore a collection of baseline algorithms for subjective extractive summarization and show that (i) as expected, example-based approaches better capture subjective intents than query-based ones, and (ii) there is ample scope for improving upon the baseline algorithms, thereby motivating further research on this challenging problem.
