Describing troubling events and images and reflecting on their emotional meanings are central components of most psychotherapies. The computer system described here tracks the occurrence and intensity of narration or imagery within transcribed therapy sessions and over the course of treatments; it likewise tracks the extent to which language denoting appraisal and logical thought occurs. The Discourse Attributes Analysis Program (DAAP) is a computer text analysis system that uses several dictionaries, including the Weighted Referential Activity Dictionary (WRAD), designed to detect verbal communication of emotional images and events, and the Reflection Dictionary (REF), designed to detect verbal communication denoting cognitive appraisal, as well as other dictionaries. For each dictionary and each turn of speech, DAAP uses a moving weighted average of dictionary weights, together with a fold-over procedure, to produce a smooth density function that graphically illustrates the rise and fall of each underlying psychological variable. These density functions are then used to produce several new measures, including measures of the vividness of descriptions of images or events, and a measure of the extent to which descriptions of events or images and reflection on their meaning occur separately.
