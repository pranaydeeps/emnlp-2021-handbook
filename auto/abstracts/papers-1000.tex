It is often challenging to solve a complex problem from scratch, but much easier if we can access other similar problems with their solutions --- a paradigm known as case-based reasoning (CBR). We propose a neuro-symbolic CBR approach (CBR-KBQA) for question answering over large knowledge bases. CBR-KBQA consists of a nonparametric memory that stores cases (question and logical forms) and a parametric model that can generate a logical form for a new question by retrieving cases that are relevant to it. On several KBQA datasets that contain complex questions, CBR-KBQA achieves competitive performance. For example, on the CWQ dataset, CBR-KBQA outperforms the current state of the art by 11\% on accuracy. Furthermore, we show that CBR-KBQA is capable of using new cases \emph{without} any further training: by incorporating a few human-labeled examples in the case memory, CBR-KBQA is able to successfully generate logical forms containing unseen KB entities as well as relations.
