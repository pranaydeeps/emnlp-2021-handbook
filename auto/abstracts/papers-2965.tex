Contextualised word embeddings is a powerful tool to detect contextual synonyms. However, most of the current state-of-the-art (SOTA) deep learning concept extraction methods remain supervised and underexploit the potential of the context. In this paper, we propose a self-supervised pre-training approach which is able to detect contextual synonyms of concepts being training on the data created by shallow matching. We apply our methodology in the sparse multi-class setting (over 15,000 concepts) to extract phenotype information from electronic health records. We further investigate data augmentation techniques to address the problem of the class sparsity. Our approach achieves a new SOTA for the unsupervised phenotype concept annotation on clinical text on F1 and Recall outperforming the previous SOTA with a gain of up to 4.5 and 4.0 absolute points, respectively. After fine-tuning with as little as 20\% of the labelled data, we also outperform BioBERT and ClinicalBERT. The extrinsic evaluation on three ICU benchmarks also shows the benefit of using the phenotypes annotated by our model as features.
