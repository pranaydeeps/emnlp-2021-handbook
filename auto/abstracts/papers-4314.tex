Answering questions asked from instructional corpora such as E-manuals, recipe books, etc., has been far less studied than open-domain factoid context-based question answering. This can be primarily attributed to the absence of standard benchmark datasets. In this paper, we meticulously create a large amount of data connected with E-manuals and develop a suitable algorithm to exploit it. We collect E-Manual Corpus, a huge corpus of 307,957 E-manuals, and pretrain RoBERTa on this large corpus. We create various benchmark QA datasets which include question answer pairs curated by experts based upon two E-manuals,  real user questions from Community Question Answering Forum pertaining to E-manuals etc. We introduce EMQAP (E-Manual Question Answering Pipeline) that answers questions pertaining to electronics devices. Built upon the pretrained RoBERTa, it harbors a  supervised multi-task learning framework which efficiently performs the dual tasks of identifying the section in the E-manual where the answer can be found and the exact answer span within that section. For E-Manual annotated question-answer pairs, we show an improvement of about 40\% in ROUGE-L F1 scores over most competitive baseline. We perform a detailed ablation study and establish the versatility of EMQAP across different circumstances. The code and datasets are shared at https://github.com/abhi1nandy2/EMNLP-2021-Findings, and the corresponding project website is https://sites.google.com/view/emanualqa/home.
