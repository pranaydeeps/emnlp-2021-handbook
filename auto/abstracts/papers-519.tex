This work proposes an extensive analysis of the Transformer architecture in the Neural Machine Translation (NMT) setting. Focusing on the encoder-decoder attention mechanism, we prove that attention weights systematically make alignment errors by relying mainly on uninformative tokens from the source sequence. However, we observe that NMT models assign attention to these tokens to regulate the contribution in the prediction of the two contexts, the source and the prefix of the target sequence. We provide evidence about the influence of wrong alignments on the model behavior, demonstrating that the encoder-decoder attention mechanism is well suited as an interpretability method for NMT. Finally, based on our analysis, we propose methods that largely reduce the word alignment error rate compared to standard induced alignments from attention weights.
