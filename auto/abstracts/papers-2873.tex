Since state-of-the-art approaches to offensive language detection rely on supervised learning, it is crucial to quickly adapt them to the continuously evolving scenario of social media. While several approaches have been proposed to tackle the problem from an algorithmic perspective, so to reduce the need for annotated data, less attention has been paid to the quality of these data. Following a trend that has emerged recently, we focus on the level of agreement among annotators while selecting data to create offensive language datasets, a task involving a high level of subjectivity. Our study comprises the creation of three novel datasets of English tweets covering different topics and having five crowd-sourced judgments each. We also present an extensive set of experiments showing that  selecting training and test data according to different levels of annotators' agreement has a strong effect on classifiers performance and robustness. Our findings are further validated in cross-domain experiments and studied using a popular benchmark dataset. We show that such hard cases, where low agreement is present, are not necessarily due to poor-quality annotation and we advocate for a higher presence of ambiguous cases in future datasets, in order to train more robust systems and better account for the different points of view expressed online.
