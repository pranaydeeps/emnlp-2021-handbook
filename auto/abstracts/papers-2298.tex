A long-standing issue with paraphrase generation is the lack of reliable supervision signals. In this paper, we propose a new unsupervised paradigm for paraphrase generation based on the assumption that the probabilities of generating two sentences with the same meaning given the same context should be the same. Inspired by this fundamental idea, we propose a pipelined system which consists of paraphrase candidate generation based on contextual language models, candidate filtering using scoring functions, and paraphrase model training based on  the selected candidates. The proposed paradigm offers merits over existing paraphrase generation methods: (1) using the context regularizer on meanings, the model is able to generate massive amounts of high-quality paraphrase pairs; (2) the combination of the huge amount of paraphrase candidates and further diversity-promoting filtering yields paraphrases with more lexical and syntactic diversity; and (3) using human-interpretable scoring functions to select paraphrase pairs from candidates, the proposed framework provides a channel for developers to intervene with the data generation process, leading to a more controllable model. Experimental results across different tasks and datasets demonstrate that the proposed paradigm significantly outperforms existing paraphrase approaches in both supervised and unsupervised setups.
