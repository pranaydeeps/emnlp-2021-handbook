The recurrent neural network (RNN) language model is a powerful tool for learning arbitrary sequential dependencies in language data. Despite its enormous success in representing lexical sequences, little is known about the quality of the lexical representations that it acquires. In this work, we conjecture that it is straightforward to extract lexical representations (i.e. static word embeddings) from an RNN, but that the amount of semantic information that is encoded is limited when lexical items in the training data provide redundant semantic information. We conceptualize this limitation of the RNN as a failure to learn atomic internal states - states which capture information relevant to single word types without being influenced by redundant information provided by words with which they co-occur. Using a corpus of artificial language, we verify that redundancy in the training data yields non-atomic internal states, and propose a novel method for inducing atomic internal states. We show that 1) our method successfully induces atomic internal organization in controlled experiments, and 2) under more realistic conditions in which the training consists of child-directed language, application of our method improves the performance of lexical representations on a downstream semantic categorization task.
