Auxiliary information from multiple sources has been demonstrated to be effective in zero-shot fine-grained entity typing (ZFET). However, there lacks a comprehensive understanding about how to make better use of the existing information sources and how they affect the performance of ZFET. In this paper, we empirically study three kinds of auxiliary information: context consistency, type hierarchy and background knowledge (e.g., prototypes and descriptions) of types, and propose a multi-source fusion model (MSF) targeting these sources. The performance obtains up to 11.42\% and 22.84\% absolute gains over state-of-the-art baselines on BBN and Wiki respectively with regard to macro F1 scores. More importantly, we further discuss the characteristics, merits and demerits of each information source and provide an intuitive understanding of the complementarity among them.
