Inflectional morphology has since long been a useful testing ground for broader questions about generalisation in language and the viability of neural network models as cognitive models of language. Here, in line with that tradition, we explore how recurrent neural networks acquire the complex German plural system and reflect upon how their strategy compares to human generalisation and rule-based models of this system. We perform analyses including behavioural experiments, diagnostic classification, representation analysis and causal interventions, suggesting that the models rely on features that are also key predictors in rule-based models of German plurals. However, the models also display shortcut learning, which is crucial to overcome in search of more cognitively plausible generalisation behaviour.
