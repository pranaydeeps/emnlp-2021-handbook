It is well known that prosodic information is used by infants in early language acquisition. In particular, prosodic boundaries have been shown to help infants with sentence and word-level segmentation. In this study, we extend an unsupervised method for word segmentation to include information about prosodic boundaries. The boundary information used was either derived from oracle data (hand-annotated), or extracted automatically with a system that employs only acoustic cues for boundary detection. The approach was tested on two different languages, English and Japanese, and the results show that boundary information helps word segmentation in both cases. The performance gain obtained for two typologically distinct languages shows the robustness of prosodic information for word segmentation. Furthermore, the improvements are not limited to the use of oracle information, similar performances being obtained also with automatically extracted boundaries.
