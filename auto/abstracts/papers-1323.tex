Most previous studies on information status (IS) classification and bridging anaphora recognition assume that the gold mention or syntactic tree information is given (Hou et al., 2013; Roesiger et al., 2018; Hou, 2020; Yu and Poesio, 2020). In this paper, we propose an end-to-end neural approach for information status classification. Our approach consists of a mention extraction component and an information status assignment component. During the inference time, our system takes a raw text as the input and generates mentions together with their information status. On the ISNotes corpus (Markert et al., 2012), we show that our information status assignment component achieves new state-of-the-art results on fine-grained IS classification based on gold mentions. Furthermore, our system performs significantly better than other baselines for both mention extraction and fine-grained IS classification in the end-to-end setting. Finally, we apply our system on BASHI (Roesiger, 2018) and SciCorp (Roesiger, 2016) to recognize referential bridging anaphora. We find that our end-to-end system trained on ISNotes achieves competitive results on bridging anaphora recognition compared to the previous state-of-the-art system that relies on syntactic information and is trained on the in-domain datasets (Yu and Poesio, 2020).
