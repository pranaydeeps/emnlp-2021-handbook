Over the past few years, Word Sense Disambiguation (WSD) has received renewed interest: recently proposed systems have shown the remarkable effectiveness of deep learning techniques in this task, especially when aided by modern pretrained language models. Unfortunately, such systems are still not available as ready-to-use end-to-end packages, making it difficult for researchers to take advantage of their performance. The only alternative for a user interested in applying WSD to downstream tasks is to rely on currently available end-to-end WSD systems, which, however, still rely on graph-based heuristics or non-neural machine learning algorithms. In this paper, we fill this gap and propose AMuSE-WSD, the first end-to-end system to offer high-quality sense information in 40 languages through a state-of-the-art neural model for WSD. We hope that AMuSE-WSD will provide a stepping stone for the integration of meaning into real-world applications and encourage further studies in lexical semantics. AMuSE-WSD is available online at http://nlp.uniroma1.it/amuse-wsd.