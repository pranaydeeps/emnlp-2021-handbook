Entity linking is an important problem with many applications. Most previous solutions were designed for settings where annotated training data is available, which is, however, not the case in numerous domains. We propose a light-weight and scalable entity linking method, Eigenthemes, that relies solely on the availability of entity names and a referent knowledge base. Eigenthemes exploits the fact that the entities that are truly mentioned in a document (the ``gold entities'') tend to form a semantically dense subset of the set of all candidate entities in the document. Geometrically speaking, when representing entities as vectors via some given embedding, the gold entities tend to lie in a low-rank subspace of the full embedding space. Eigenthemes identifies this subspace using the singular value decomposition and scores candidate entities according to their proximity to the subspace. On the empirical front, we introduce multiple strong baselines that compare favorably to (and sometimes even outperform) the existing state of the art. Extensive experiments on benchmark datasets from a variety of real-world domains showcase the effectiveness of our approach.
