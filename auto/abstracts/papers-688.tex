With the emergence of the COVID-19 pandemic, the political and the medical aspects of disinformation merged as the problem got elevated to a whole new level to become the first global infodemic. Fighting this infodemic has been declared one of the most important focus areas of the World Health Organization, with dangers ranging from promoting fake cures, rumors, and conspiracy theories to spreading xenophobia and panic. Addressing the issue requires solving a number of challenging problems such as identifying messages containing claims, determining their check-worthiness and factuality, and their potential to do harm as well as the nature of that harm, to mention just a few. To address this gap, we release a large dataset of 16K manually annotated tweets for fine-grained disinformation analysis that (i) focuses on COVID-19, (ii) combines the perspectives and the interests of journalists, fact-checkers, social media platforms, policy makers, and society, and (iii) covers Arabic, Bulgarian, Dutch, and English. Finally, we show strong evaluation results using pretrained Transformers, thus confirming the practical utility of the dataset in monolingual vs. multilingual, and single task vs. multitask settings.
