In this article, we tackle the math word problem, namely, automatically answering a mathematical problem according to its textual description. Although recent methods have demonstrated their promising results, most of these methods are based on template-based generation scheme which results in limited generalization capability. To this end, we propose a novel human-like analogical learning method in a recall and learn manner. Our proposed framework is composed of modules of memory, representation, analogy, and reasoning, which are designed to make a new exercise by referring to the exercises learned in the past. Specifically, given a math word problem, the model first retrieves similar questions by a memory module and then encodes the unsolved problem and each retrieved question using a representation module. Moreover, to solve the problem in a way of analogy, an analogy module and a reasoning module with a copy mechanism are proposed to model the interrelationship between the problem and each retrieved question. Extensive experiments on two well-known datasets show the superiority of our proposed algorithm as compared to other state-of-the-art competitors from both overall performance comparison and micro-scope studies.
