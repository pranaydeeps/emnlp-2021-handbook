Language models trained on billions of tokens have recently led to unprecedented results on many NLP tasks. This success raises the question of whether, in principle, a system can ever "understand" raw text without access to some form of grounding. We formally investigate the abilities of ungrounded systems to acquire meaning. Our analysis focuses on the role of "assertions": textual contexts that provide indirect clues about the underlying semantics. We study whether assertions enable a system to emulate representations preserving semantic relations like equivalence. We find that assertions enable semantic emulation of languages that satisfy a strong notion of semantic transparency. However, for classes of languages where the same expression can take different values in different contexts, we show that emulation can become uncomputable. Finally, we discuss differences between our formal model and natural language, exploring how our results generalize to a modal setting and other semantic relations. Together, our results suggest that assertions in code or language do not provide sufficient signal to fully emulate semantic representations. We formalize ways in which ungrounded language models appear to be fundamentally limited in their ability to "understand".