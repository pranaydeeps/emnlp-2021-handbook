Loading models pre-trained on the large-scale corpus in the general domain and fine-tuning them on specific downstream tasks is gradually becoming a paradigm in Natural Language Processing. Previous investigations prove that introducing a further pre-training phase between pre-training and fine-tuning phases to adapt the model on the domain-specific unlabeled data can bring positive effects. However, most of these further pre-training works just keep running the conventional pre-training task, e.g., masked language model, which can be regarded as the domain adaptation to bridge the data distribution gap. After observing diverse downstream tasks, we suggest that different tasks may also need a further pre-training phase with appropriate training tasks to bridge the task formulation gap. To investigate this, we carry out a study for improving multiple task-oriented dialogue downstream tasks through designing various tasks at the further pre-training phase. The experiment shows that different downstream tasks prefer different further pre-training tasks, which have intrinsic correlation and most further pre-training tasks significantly improve certain target tasks rather than all. Our investigation indicates that it is of great importance and effectiveness to design appropriate further pre-training tasks modeling specific information that benefit downstream tasks. Besides, we present multiple constructive empirical conclusions for enhancing task-oriented dialogues.
