Natural language relies on a finite lexicon to express an unbounded set of emerging ideas. One result of this tension is the formation of new compositions, such that existing linguistic units can be combined with emerging items into novel expressions. We develop a framework that exploits the cognitive mechanisms of chaining and multimodal knowledge to predict emergent compositional expressions through time. We present the syntactic frame extension model (SFEM) that draws on the theory of chaining and knowledge from ``percept'', ``concept'', and ``language'' to infer how verbs extend their frames to form new compositions with existing and novel nouns. We evaluate SFEM rigorously on the 1) modalities of knowledge and 2) categorization models of chaining, in a syntactically parsed  English corpus over the past 150 years. We show that multimodal SFEM predicts newly emerged verb syntax and arguments substantially better than competing models using purely linguistic or unimodal knowledge. We  find support for an exemplar view of chaining as opposed to a prototype view and reveal how the joint approach of multimodal chaining may be fundamental to the creation of literal and figurative language uses including metaphor and metonymy.
