Unsupervised Data Augmentation (UDA) is a semisupervised technique that applies a consistency loss to penalize differences between a model's predictions on (a) observed (unlabeled) examples; and (b) corresponding 'noised' examples produced via data augmentation. While UDA has gained popularity for text classification, open questions linger over which design decisions are necessary and how to extend the method to sequence labeling tasks. In this paper, we re-examine UDA and demonstrate its efficacy on several sequential tasks. Our main contribution is an empirical study of UDA to establish which components of the algorithm confer benefits in NLP. Notably, although prior work has emphasized the use of clever augmentation techniques including back-translation, we find that enforcing consistency between predictions assigned to observed and randomly substituted words often yields comparable (or greater) benefits compared to these more complex perturbation models. Furthermore, we find that applying UDA's consistency loss affords meaningful gains without any unlabeled data at all, i.e., in a standard supervised setting. In short, UDA need not be unsupervised to realize much of its noted benefits, and does not require complex data augmentation to be effective.
