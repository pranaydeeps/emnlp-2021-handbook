This paper presents a new corpus and annotation guideline for a novel coreference resolution task on fictional texts, and analyzes its unique characteristics. FantasyCoref contains 211 stories of Grimms' Fairy Tales and 3 other fantasy literature annotated in the omniscient writer's point of view (OWV) to handle distinctive aspects in this genre. This task is more challenging than general coreference resolution in two ways. First, documents in our corpus are 2.5 times longer than the ones in OntoNotes, raising a new layer of difficulty in resolving long-distant referents. Second, annotation of literary styles and concepts raise several issues which are not sufficiently addressed in the existing annotation guidelines. Hence, considerations on such issues and the concept of OWV are necessary to achieve high inter-annotator agreement (IAA) in coreference resolution of fictional texts. We carefully conduct annotation tasks in four stages to ensure the quality of our annotation. As a result, a high IAA score of 87\% is achieved using the standard coreference evaluation metric. Finally, state-of-the-art coreference resolution approaches are evaluated on our corpus. After training with our annotated dataset, there was a 2.59\% and 3.06\% improvement over the model trained on the OntoNotes dataset. Also, we observe that the portion of errors specific to fictional texts declines after the training.
