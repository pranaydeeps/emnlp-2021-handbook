Individuals signal aspects of their identity and beliefs through linguistic choices. Studying these choices in aggregate allows us to examine large-scale attitude shifts within a population. Here, we develop computational methods to study word choice within a sociolinguistic lexical variable—alternate words used to express the same concept—in order to test for change in the United States towards sexuality and gender. We examine two variables: i) referents to significant others, such as the word ``partner'' and ii) referents to an indefinite person, both of which could optionally be marked with gender. The linguistic choices in each variable allow us to study increased rates of acceptances of gay marriage and gender equality, respectively. In longitudinal analyses across Twitter and Reddit over 87M messages, we demonstrate that attitudes are changing but that these changes are driven by specific demographics within the United States. Further, in a quasi-causal analysis, we show that passages of Marriage Equality Acts in different states are drivers of linguistic change.
