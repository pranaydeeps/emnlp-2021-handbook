Literature search is the most crucial part of scientific work in NLP. It plays a vital role in devel- oping an understanding of a research problem, finding the previously explored frontiers, identi- fying research gaps, and leading to the development of new ideas. However, with the exponential growth of scientific literature (including published papers and pre-prints), it is almost impossible for a researcher to go through the entire body of the scholarly works even in a very narrow do- main. As a result, researchers spend a lot of their time searching for the most relevant papers to their research topic and locating the literature that carried forward a given scientific idea. Also, it is usually desirable for a researcher to understand the story behind a prior work, trace the emer- gence of the concept, and its gradual evolution through publications and identify the knowledge flow.
