The paper reports the results of a translationese study of literary texts based on translated and non-translated Russian. We aim to find out if translations deviate from non-translated literary texts, and if the established differences can be attributed to typological relations between source and target languages. We expect that literary translations from typologically distant languages should exhibit more translationese, and the fingerprints of individual source languages (and their families) are traceable in translations. We explore linguistic properties that distinguish non-translated Russian literature from translations into Russian. Our results show that non-translated fiction is different from translations to the degree that these two language varieties can be automatically classified. As expected, language typology is reflected in translations of literary texts. We identified features that point to linguistic specificity of Russian  non-translated literature and to shining-through effects. Some of translationese features cut across all language pairs, while others are characteristic of literary translations from languages belonging to specific language families.
