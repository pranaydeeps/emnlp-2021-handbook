Law, interpretations of law, legal arguments, agreements, etc. are typically expressed in writing, leading to the production of vast corpora of legal text. Their analysis, which is at the center of legal practice, becomes increasingly elaborate as these collections grow in size. Natural language understanding (NLU) technologies can be a valuable tool to support legal practitioners in these endeavors. Their usefulness, however, largely depends on whether current state-of-the-art models can generalize across various tasks in the legal domain. To answer this currently open question, we introduce the Legal General Language Understanding Evaluation (LexGLUE) benchmark, a collection of datasets for evaluating model performance across a diverse set of legal NLU tasks in a standardized way. We also provide an evaluation and analysis of several generic and legal-oriented models demonstrating that the latter consistently offer performance improvements across multiple tasks.
