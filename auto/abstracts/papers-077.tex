The factual knowledge acquired during pre-training and stored in the parameters of Language Models (LMs) can be useful in downstream tasks (e.g., question answering or textual inference). However, some facts can be incorrectly induced or become obsolete over time. We present KnowledgeEditor, a method which can be used to edit this knowledge and, thus, fix 'bugs' or unexpected predictions without the need for expensive re-training or fine-tuning. Besides being computationally efficient, KnowledgeEditordoes not require any modifications in LM pre-training (e.g., the use of meta-learning). In our approach, we train a hyper-network with constrained optimization to modify a fact without affecting the rest of the knowledge; the trained hyper-network is then used to predict the weight update at test time. We show KnowledgeEditor's efficacy with two popular architectures and knowledge-intensive tasks: i) a BERT model fine-tuned for fact-checking, and ii) a sequence-to-sequence BART model for question answering. With our method, changing a prediction on the specific wording of a query tends to result in a consistent change in predictions also for its paraphrases. We show that this can be further encouraged by exploiting (e.g., automatically-generated) paraphrases during training. Interestingly, our hyper-network can be regarded as a 'probe' revealing which components need to be changed to manipulate factual knowledge; our analysis shows that the updates tend to be concentrated on a small subset of components. Source code available at https://github.com/nicola-decao/KnowledgeEditor
