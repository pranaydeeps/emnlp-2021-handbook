Much of the existing linguistic data in many languages of the world is locked away in non-digitized books and documents. Optical character recognition (OCR) can be used to produce digitized text, and previous work has demonstrated the utility of neural post-correction methods that improve the results of general-purpose OCR systems on recognition of less-well-resourced languages. However, these methods rely on manually curated post-correction data, which are relatively scarce compared to the non-annotated raw images that need to be digitized. In this paper, we present a semi-supervised learning method that makes it possible to utilize these raw images to improve performance, specifically through the use of self-training, a technique where a model is iteratively trained on its own outputs. In addition, to enforce consistency in the recognized vocabulary, we introduce a lexically-aware decoding method that augments the neural post-correction model with a count-based language model constructed from the recognized texts, implemented using weighted finite-state automata (WFSA) for efficient and effective decoding. Results on four endangered languages demonstrate the utility of the proposed method, with relative error reductions of 15-29\%, where we find the combination of self-training and lexically-aware decoding essential for achieving consistent improvements.