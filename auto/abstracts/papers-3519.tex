Fine-grained control of machine translation (MT) outputs along multiple attributes is critical for many modern MT applications and is a requirement for gaining users' trust. A standard approach for exerting control in MT is to prepend the input with a special tag to signal the desired output attribute. Despite its simplicity, attribute tagging has several drawbacks: continuous values must be binned into discrete categories, which is unnatural for certain applications; interference between multiple tags is poorly understood. We address these problems by introducing vector-valued interventions which allow for fine-grained control over multiple attributes simultaneously via a weighted linear combination of the corresponding vectors. For some attributes, our approach even allows for fine-tuning a model trained without annotations to support such interventions. In experiments with three attributes (length, politeness and monotonicity) and two language pairs (English to German and Japanese) our models achieve better control over a wider range of tasks compared to tagging, and translation quality does not degrade when no control is requested. Finally, we demonstrate how to enable control in an already trained model after a relatively cheap fine-tuning stage.
