Extracting salient topics from a collection of documents can be a challenging task when a) the amount of data is large, b) the number of topics is not known a priori, and/or c) ``topic noise'' is present. We define ``topic noise'' as the collection of documents that are irrelevant to any coherent topic and should be filtered out. By design, most clustering algorithms (e.g. k-means, hierarchical clustering) assign all input documents to one of the available clusters, guaranteeing any topic noise to propagate into the result. To address these challenges, we present a novel algorithm, {FANATIC}, that efficiently distinguishes documents from genuine topics and those that are topic noise. We also introduce a new {R}eddit dataset to showcase {FANATIC} as it contains short, noisy data that is difficult to cluster using most clustering algorithms. We find that {FANATIC} clusters 500k {R}eddit titles (of which 20\% are topic noise) in 2 minutes and achieves an {AMI} score of 0.59, in contrast with hdbscan (McInnes et al., 2017), a popular algorithm suited for this type of task, which requires over 7 hours and achieves an {AMI} of 0.03. Finally, we test {FANATIC} against a {T}witter dataset and find again that it outperforms the other algorithms with an {AMI} score of 0.60. We make our code and data publicly available.
