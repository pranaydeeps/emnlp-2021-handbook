Practical dialogue systems require robust methods of detecting out-of-scope (OOS) utterances to avoid conversational breakdowns and related failure modes. Directly training a model with labeled OOS examples yields reasonable performance, but obtaining such data is a resource-intensive process.  To tackle this limited-data problem, previous methods focus on better modeling the distribution of in-scope (INS) examples. We introduce GOLD as an orthogonal technique that augments existing data to train better OOS detectors operating in low-data regimes. GOLD generates pseudo-labeled candidates using samples from an auxiliary dataset and keeps only the most beneficial candidates for training through a novel filtering mechanism. In experiments across three target benchmarks, the top GOLD model outperforms all existing methods on all key metrics, achieving relative gains of 52.4\%, 48.9\% and 50.3\% against median baseline performance. We also analyze the unique properties of OOS data to identify key factors for optimally applying our proposed method.
