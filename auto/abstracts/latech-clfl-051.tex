In this study, we study language change in Chinese Biji by using a classification task: classifying Ancient Chinese texts by time periods. Specifically, we focus on a unique genre in classical Chinese literature: Biji (literally ``notebook'' or ``brush notes''), i.e., collections of anecdotes, quotations, etc., anything authors consider noteworthy, Biji span hundreds of years across many dynasties and conserve informal language in written form. For these reasons, they are regarded as a good resource for investigating language change in Chinese (Fang, 2010). In this paper, we create a new dataset of 108 Biji across four dynasties. Based on the dataset, we first introduce a time period classification task for Chinese. Then we investigate different feature representation methods for classification. The results show that models using contextualized embeddings perform best. An analysis of the top features chosen by the word n-gram model (after bleaching proper nouns) confirms that these features are informative and correspond to observations and assumptions made by historical linguists.
