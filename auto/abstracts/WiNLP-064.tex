The global growth in the use of social media over the past decade has opened up opportunities for the research community to mine health-related information for tasks centered mostly around pharmacovigilance and patient-centered outcomes. Understanding user-generated posts about drug use is a challenging task, often requiring non-literal interpretation. In this paper, we focus on the classification of tweets as sources of potential signals for adverse drug effects (ADEs) or drug reactions (ADRs). Following the intuition that text and drug structure representations are complementary, we introduce a multimodal model with two components. These components are state-of-the-art BERT-based models for language understanding and molecular property prediction. Experiments were carried out on multilingual benchmarks of the Social Media Mining for Health Research and Applications (\\#SMM4H) initiative. Our models obtained state-of-the-art results of 0.61 F\$\_1\$-measure and 0.57 F\$\_1\$-measure on \\#SMM4H 2021 Shared Tasks 1a and 2 in English and Russian, respectively. On the classification of French tweets from SMM4H 2020 Task 1, our approach pushes the state of the art by an absolute gain of 8\% F\$\_1\$. Our experiments show that the molecular information obtained from neural networks is more beneficial for ADE classification than traditional molecular descriptors.
