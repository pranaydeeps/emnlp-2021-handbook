\begin{bio}
  {\bfseries Hai Zhao} is a professor at the Department of Computer Science and Engineering, Shanghai Jiao Tong University, China. His research interest is natural language processing. He has published more than 120 papers in ACL, EMNLP, COLING, ICLR, AAAI, IJCAI, and IEEE TKDE/TASLP. He won the first place in several NLP shared tasks, such as CoNLL and SIGHAN Bakeoff and top ranking in remarkable machine reading comprehension task leaderboards such as SQuAD2.0 and RACE.
He has taught the course ``natural language processing'' in SJTU for more than 10 years. He is ACL-2017 area chair on parsing, and ACL- 2018/2019 (senior) area chairs on morphology and word segmentation.

  {\bfseries Rui Wang} is a tenured researcher at the Advanced Translation Technology Laboratory, National Institute of Information and Communications Technology (NICT), Japan. His research focuses on machine translation (MT), a classic task in NLP. His recent interests are traditional linguistic based and cutting-edge machine learning based approaches for MT. He (as the first or the corresponding authors) has published more than 30 MT papers in top-tier NLP/ML/AI conferences and journals, such as ACL, EMNLP, ICLR, AAAI, IJCAI, IEEE/ACM transactions, etc. He has also won several first places in top-tier MT shared tasks, such as WMT- 2018, WMT-2019, WMT-2020, etc. He has given several tutorial and invited talks in conferences, such as CWMT, CCL, etc. He served as the area chairs of ICLR-2021 and NAACL- 2021.

  {\bfseries Kehai Chen} is a postdoctoral researcher at the Advanced Translation Technology Laboratory, National Institute of Information and Communications Technology (NICT), Japan. His research focuses on linguistic-motivated machine translation (MT), a classic NLP task in AI. He has published more than 20 MT and NLP papers in top-tier NLP/ML/AI conferences and journals, such as ACL, ICLR, AAAI, EMNLP, IEEE/ACM Transactions on Audio, Speech, and Language Processing, ACM Transactions on Asian and Low-Resource Language Information Processing, etc. He served as a senior program committee of AAAI-2021.

\end{bio}

\begin{tutorial}
  {Syntax in End-to-End Natural Language Processing}
  {tutorial-final-06}
  {\daydateyear, \tutorialmorningtime}
  {\TutLocA}

This tutorial surveys the latest technical progress of syntactic parsing and the role of syntax in end-to-end natural language processing (NLP) tasks, in which semantic role labeling (SRL) and machine translation (MT) are the representative NLP tasks that have always been beneficial from informative syntactic clues since a long time ago, though the advance from end-to-end deep learning models shows new results. In this tutorial, we will first introduce the background and the latest progress of syntactic parsing and SRL/NMT. Then, we will summarize the key evidence about the syntactic impacts over these two concerning tasks, and explore the behind reasons from both computational and linguistic background.

\end{tutorial}
