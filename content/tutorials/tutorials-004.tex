\begin{bio}
  {\bfseries Sebastian Ruder} is a research scientist at DeepMind where he works on transfer learning and multilingual natural language processing. He has been area chair in machine learning and multilinguality for major NLP conferences including ACL and EMNLP and has published papers on multilingual question answering (Artetxe et al., 2020; Hu et al., 2020). He was the Co-Program Chair for EurNLP 2019 and has co-organized the 4th Workshop on Representation Learning for NLP at ACL 2019. He has taught tutorials on ``Transfer learning in natural language processing'' and ``Unsupervised Cross-lingual Representation Learning'' at NAACL 2019 and ACL 2019 respectively. He has also co-organized and taught at the NLP Session at the Deep Learning Indaba 2018 and 2019.

  {\bfseries Avirup Sil} is a Research Scientist and the Team Lead for Question Answering in the Multilingual NLP group at IBM Research AI. His team (comprising of research scientists and engineers) works on research on industry scale NLP and Deep Learning algorithms. His teamÕs system called `GAAMA' has obtained the top scores in public benchmark datasets (Kwiatkowski et al., 2019) and has published several papers on question answering (Chakravarti et al., 2019; Castelli et al., 2020; Glass et al., 2020). He is also the Chair of the NLP professional community of IBM. Avi is a Senior Program Committee Member and the Area Chair in Question Answering for major NLP conferences e.g. ACL, EMNLP, NAACL and has published several papers on Question Answering. He has taught a tutorial at ACL 2018 on ``Entity Discovery and Linking''. He has also organized the workshop on the ``Relevance of Linguistic Structure in Neural NLP'' at ACL 2018. He is also the track coordinator for the Entity Discovery and Linking track at the Text Analysis Conference.

\end{bio}

\begin{tutorial}
  {Multi-Domain Multilingual Question Answering}
  {Sebastian Ruder, Avirup Sil}
  {\daydateyear, \tutorialmorningtime}


Question answering (QA) is one of the most challenging and impactful tasks in natural language processing. Most research in QA and tutorials, however, has focused on the open-domain or monolingual setting while most real-world applications deal with specific domains or languages. In this tutorial, we attempt to bridge this gap. Firstly, we introduce standard benchmarks in multi-domain and multilingual QA. In both scenarios, we discuss state-of-the-art approaches that achieve impressive performance by either zero-shot learning or out-of-the-box training on open (and closed)-domain QA systems. Finally, we will present open research problems that this new research agenda poses such as multi-task learning, cross-lingual transfer learning, domain adaptation and training large scale pre-trained multilingual language models.

\end{tutorial}
