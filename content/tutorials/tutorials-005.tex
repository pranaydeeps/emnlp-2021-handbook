\begin{bio}
  {\bfseries Kai-Wei Chang} is an assistant professor in the Department of Computer Science at the University of California Los Angeles. His research interests include designing robust, fair, and accountable machine learning methods for building reliable NLP systems (e.g., Alzantot et al., 2018; Shi et al., 2019). His awards include the EMNLP Best Long Paper Award (2017), the KDD Best Paper Award (2010), and the Sloan Research Fellowship (2021). Kai-Wei has given tutorials at NAACL 15, AAAI 16, FAccT18, EMNLP 19, AAAI 20, MLSS 21 on different research topics. Additional information is available at http://kwchang.net.

  {\bfseries He He} is an assistant professor in the Department of Computer Science and the Center for Data Science at the New York University. Her research interests include reliable natural language generation and robust learning algorithms that avoid spurious correlations in the data (e.g., He et al., 2019; Tu et al., 2020). She has given tutorials at NAACL 15 and EMNLP 19. Additional information is available at http://hhexiy.github.io.

  {\bfseries Robin Jia} is currently a visiting researcher at Facebook AI Research, and will be an assistant professor in the Department of Computer Science at the University of Southern California starting in the Autumn of 2021. His research focuses on making natural language processing models robust to unexpected test-time distribution shifts (e.g., Jia and Liang, 2017; Jia et al., 2019). RobinÕs work has received an Outstanding Paper Award at EMNLP 2017 and a Best Short Paper Award at ACL 2018. Additional information is available at https://robinjia.github.io.

  {\bfseries Sameer Singh} is an Assistant Professor of Computer Science at the University of California, Irvine. He is working on large-scale and interpretable machine learning models for NLP (e.g., Wallace et al., 2019a; Pezeshkpour et al., 2019). His work has received paper awards at ACL 2020, AKBC 2020, EMNLP 2019, ACL 2018, and KDD 2016. Sameer presented the Deep Adversarial Learning Tutorial (Wang et al., 2019) at NAACL 2019 and the Mining Knowledge Graphs from Text Tutorial at WSDM 2018 and AAAI 2017, along with tutorials on Interpretability and Explanations in upcoming NeurIPS 2020 and EMNLP 2020. Sameer has also received teaching awards at UCI. Website: http://sameersingh.org/.

\end{bio}

\begin{tutorial}
  {Robustness and Adversarial Examples in Natural Language Processing}
  {tutorial-final-05}
  {\daydateyear, \tutorialmorningtime}
  {\TutLocA}

Recent studies show that many NLP systems are sensitive and vulnerable to a small perturbation of inputs and do not generalize well across different datasets. This lack of robustness derails the use of NLP systems in real-world applications. This tutorial aims at bringing awareness of practical concerns about NLP robustness. It targets NLP researchers and practitioners who are interested in building reliable NLP systems. In particular, we will review recent studies on analyzing the weakness of NLP systems when facing adversarial inputs and data with a distribution shift. We will provide the audience with a holistic view of 1) how to use adversarial examples to examine the weakness of NLP models and facilitate debugging; 2) how to enhance the robustness of existing NLP models and defense against adversarial inputs; and 3) how the consideration of robustness affects the real-world NLP applications used in our daily lives. We will conclude the tutorial by outlining future research directions in this area.

\end{tutorial}
