\begin{bio}
  {\bfseries Wenhao Yu} is a Ph.D. student in the Department of Computer Science and Engineering at the University of Notre Dame. His research lies in controllable knowledge-driven natural language processing, particularly in natural language generation. His research has been published in top-ranked NLP and data mining conferences such as ACL, EMNLP, AAAI, WWW, and CIKM. Additional information is available at https://wyu97.github.io/.

  {\bfseries Meng Jiang} is an assistant professor in the Department of Computer Science and Engineering at the University of Notre Dame. He received his B.E. and Ph.D. in Computer Science from Tsinghua University and was a postdoctoral research associate at the University of Illinois at Urbana-Champaign. His research interests focus on knowledge graph
construction and natural language generation for news summarization and forum post generation. The awards he received include Notre Dame Faculty Award in 2019 and Best Paper Awards at ISDSA and KDD-DLG in 2020. Additional information is available at http://www.meng-jiang.com/.

  {\bfseries Zhiting Hu} is an assistant professor in Halicio\v{g}lu Data Science Institute at UC San Diego. He received his Ph.D. in Machine Learning from Carnegie Mellon University. His research interest lies in the broad area of natural language processing in particular controllable text generation, machine learning to enable training AI agents from all forms of experiences such as structured knowledge, ML systems and applications. His research was recognized with best demo nomination at ACL 2019 and outstanding paper award at ACL 2016. Additional information is available at http://www.cs.cmu.edu/?zhitingh/.

  {\bfseries Qingyun Wang} is a Ph.D. student in the Computer Science Department at the University of Illinois at Urbana-Champaign. His research lies in controllable knowledge-driven natural language generation, with a recent focus on the scientific paper generation. He served as a program committee in generation track for multiple conferences including ICML 2020, ACL 2019-2020, ICLR 2021, etc. He previously entered the finalist of the first Alexa Prize competition. Additional information is available at https://eaglew.github.io/.

  {\bfseries Heng Ji} is a professor at Computer Science Department of University of Illinois at Urbana-Champaign, and Amazon Scholar. She has published on Multimedia Multilingual Information Extraction and Knowledge-enriched NLG including technical paper generation, knowledge base description, and knowledge-aware image and video caption generation. The awards she received include ``Young Scientist''  by World Economic Fo- rum, ``AIÕs 10 to Watch'' Award by IEEE Intel- ligent Systems, NSF CAREER award, and ACL 2020 Best Demo Award. She has served as the Program Committee Co-Chair of many con- ferences including NAACL-HLT2018, and she is NAACL secretary 2020-2021. Additional information is available at https://blender.cs. illinois.edu/hengji.html.

  {\bfseries Nazneen Rajani} is a senior research scientist at Salesforce Research. She got her PhD in Computer Science from UT Austin in 2018. Several of her work has been published in ACL, EMNLP, NACCL, and IJCAI including work on generating explanations for commonsense and physical reasoning. Nazneen was one of the finalists for the VentureBeat Transform 2020 women in AI Research. Her work has been covered by several media outlets including Quanta Magazine, VentureBeat, SiliconAngle, ZDNet. More information on https://www.nazneenrajani.com.

\end{bio}

\begin{tutorial}
  {Knowledge-Enriched Natural Language Generation}
  {tutorial-final-03}
  {\daydateyear, \tutorialmorningtime}
  {\TutLocA}

Knowledge-enriched text generation poses unique challenges in modeling and learning, driving active research in several core directions, ranging from integrated modeling of neural representations and symbolic information in the sequential/hierarchical/graphical structures, learning without direct supervisions due to the cost of structured annotation, efficient optimization and inference with massive and global constraints, to language grounding on multiple modalities, and generative reasoning with implicit commonsense knowledge and background knowledge. In this tutorial we will present a roadmap to line up the state-of-the-art methods to tackle these challenges on this cutting-edge problem. We will dive deep into various technical components: how to represent knowledge, how to feed knowledge into a generation model, how to evaluate generation results, and what are the remaining challenges?

\end{tutorial}
