%\thispagestyle{myheadings}
\section{Keynote Address: Steven Bird}
\index{Bird, Steven}

\begin{center}
\begin{Large}
{\bfseries\Large LT4All!? Rethinking the Agenda}
\vspace{1em}\par
\end{Large}

\daydateyear, 9:00--10:10am \vspace{1em}\\
\PlenaryLoc \\
\vspace{1em}\par
%\includegraphics[height=100px]{content/tuesday/popovic-headshot.jpg}
\end{center}

\noindent
{\bfseries Abstract:} The majority of the world's languages are oral, emergent, untranslatable, and tightly coupled to a place. Yet it seems that the agenda is to supply all languages with the technologies that have been developed for written languages. It is as though standardised writing were the optimal way to safeguard the future of any language. It is as though the function of a language is exclusively for transmitting information, and that the same information can be rendered into any language. It is as though we can capture and model language data independently of people, purpose, and place. What would it be like if language technologies respected the self-determination of a local speech community and supported aspirations concerning the local repertoire of speech varieties? The answer will be different in different places, but there may be value in taking a close look at an individual community and trying to discern broader themes. In this talk I will share from my experience of living and working in a remote Aboriginal community in the far north of Australia. Here, local people have been teaching me participatory, relational, strengths-based approaches that my students and I have been exploring in the design of language technologies. I will reflect on five years of personal experiences in this space and share thoughts concerning an agenda for language technology in the interest of minority speech communities, and hopes for creating a world that sustains its languages.

\vspace{3em}\par

\vfill
\noindent

{\bfseries Biography:}
Steven Bird has spent 25 years pursuing scalable computational methods for capturing, enriching, and analysing data from endangered languages, drawing on fieldwork in West Africa, South America, and Melanesia. Over the past 5 years he has begun to work with remote Aboriginal communities in northern Australia. Steven has held academic positions at U Edinburgh, U Pennsylvania, UC Berkeley, and U Melbourne. He currently holds the positions of professor at Charles Darwin University, linguist at Nawarddeken Academy, and producer at languageparty.org.

\newpage
