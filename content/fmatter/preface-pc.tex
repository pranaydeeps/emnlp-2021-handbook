\section{Message from the Program Committee Co-Chairs}
\setheaders%
    {Message from the Program Committee Co-Chairs}%
    {Message from the Program Committee Co-Chairs}
\thispagestyle{emptyheader}

\setlength{\parskip}{.7ex}


Welcome to the EMNLP 2021, the first hybrid conference in EMNLP's history, which is to be held online and in Punta Cana, Dominican Republic.

EMNLP 2021 has received 3,717 full paper submissions, the largest number to date. After excluding papers withdrawn by the authors, and desk rejecting papers which violated the anonymity policy, the multiple submission policy, or the formatting requirements, we were left with 3,600 submissions to be sent out for review. Despite the record-breaking number of submissions, we were able to keep the acceptance rates at a similar level as past years. 841 submissions were accepted to the main conference. Among them, 315 were accepted as oral papers, and 526 were accepted as posters. The decision between oral and poster presentations was not based on the quality/merit of the papers, but on our understanding of what would be the best format for presentation of each individual paper.

We continued providing the acceptance option of ``Findings'', following last year's initiative in the form of a companion publication, for papers that narrowly missed acceptance to the main conference, but were judged to be solid, well-executed research, and worthy of publication. After the review process, 445 papers were invited to be included in the Findings. 26 papers declined the offer, leading to 419 papers to be published in the Findings. Some statistics of the accepted papers are shown below.


\begin{table}[h]
    \begin{center}
\begin{tabular}[h]{|c|c|c|c|}
\hline
    &Long & Short & Total \\
\hline
    Reviewed & 2,540 &1,060  &3,600   \\
\hline
    Accepted as Oral & 249& 66 & 315   \\
\hline
    Accepted as Poster & 402 & 124 & 526  \\
   \hline
    Acceptance Rate (Main Conference) & 25.6\% & 17.9\% & 23.4\%  \\
\hline
   Accepted to Findings & 300 & 119 & 419  \\
 \hline
    Acceptance Rate (Findings) & 11.8\% & 11.3\% & 11.6\%  \\
\hline
\end{tabular}
    \end{center}
\end{table}

To meet the reviewer demands of a large conference, we organized the program committee into 22 tracks, including a special ``Multidisciplinary and Area Chair Conflict of Interest'' track, based on the track information in past conferences. We also introduced a new track called ``Efficient methods for NLP'' to promote work aiming to reduce the costs of NLP design and experimentation, similar to the ``Green NLP'' tracks in EACL 2021 and NAACL 2021. In terms of submissions per track, 9 tracks received more than 200 submissions. Particularly popular were the tracks NLP Applications, Machine Learning for NLP, Machine Translation and Information Extraction, which have around 300 submissions each.

We adopted a hierarchical program committee structure similar to that of recent NLP conferences. For each area, we invited 1-4 Senior Area Chair (SACs), who worked with a team of Area Chairs (ACs) they nominated, as well as an army of reviewers that we put together. We used the submission numbers per track from past conferences to estimate the number of SACs and ACs required for each track, leading to 46 SACs and 237 ACs. For reviewer recruitment, we started with the reviewer lists from past conferences and sent out initial invitations asking reviewers to express their track preferences. We then passed the reviewer list to SACs and asked them to select reviewers from these candidate reviewers based on their expertise, and Semantic/Google Scholar profiles. Overall, this resulted in a total of 3,112 reviewers.

Each submission was assigned to three reviewers and one AC. The initial paper assignment was first made using an automatic algorithm to match the abstracts with ACs/reviewers' past publication records, then adjusted by SACs/PCs. We adapted the review forms from EMNLP 2020, NAACL 2021, and ACL-IJCNLP 2021. Besides the overall recommendation, reviewers were asked to evaluate how reproducible the results in the paper were, and whether there was any ethical concern. Our final decisions were made not just on the review scores, but also took into account the reviews, author responses, discussions among reviewers, meta-reviews and S(AC) recommendations. To ensure the review quality, we provided detailed guidelines about what reviewers should and shouldn't do in a review.

We also formed an Ethics Committee (EC) dedicated to ethical issues. 203 papers with ethical concerns raised by the technical reviewing committee were sent to the EC. The EC chairs went over the papers to determine whether a full EC review would be required. If so, the paper received one or two ethics reviews from additional reviewers recruited by the EC chairs. For any paper that was recommended to be accepted based on technical reviews and that had been referred to the EC, the EC chairs recommended one of the following to the PC chairs: (a) accept (12 EMNLP, 11 Findings), (b) conditionally accept (the ethical issues must be addressed in the camera-ready version; 17 EMNLP, 20 Findings), and (c) reject due to ethical issues (1 paper). The authors of all conditionally accepted papers (except 1 paper declining the Findings offer) submitted the camera-ready version and a short response that explained how they had made the changes requested by the EC meta-reviews. The EC chairs double-checked these revised submissions and responses, and confirmed that the ethical concerns had been addressed. As a result, all conditionally accepted papers were accepted to the main conference or Findings.

ACL Rolling Review (ARR) is a new initiative of the Association for Computational Linguistics, where the reviewing and acceptance of papers to publication venues is done in a two-step process: (1) centralized rolling review and (2) submission to a publication venue. Working closely with the ARR organizers, we ran a pilot at EMNLP 2021. 17 papers (16 long, 1 short) were submitted via ARR to EMNLP 2021, accounting for 25\% of the ARR May submissions. After the decision process involving only PCs and SACs, 6 papers (5 long, 1 short) were accepted to the main conference, among which 2 papers were accepted orally. The other 5 long papers were accepted to the Findings. These papers will be published in the respective proceedings as any other EMNLP/Findings paper.

Based on the nominations from SACs and ACs, we identified 21 candidates for the best papers and outstanding papers award. These papers are assessed by the Best Paper Award Committee. The award winners will be announced at the closing ceremony.

EMNLP 2021 will also feature 28 papers accepted by the Transactions of the Association for Computational Linguistics (TACL) and 7 papers from the journal of Computational Linguistics (CL), out of which 29 will be presented as orals and 6 as posters.

Another highlight of our program is the three exciting keynote talks, presented by Professor Ido Dagan from Bar-Ilan University, entitled ``Where next? Towards multi-text consumption via three inspired research lines'', Professor Steven Bird from Charles Darwin University, entitled ``LT4All!? Rethinking the Agenda'', and Professor Evelina Fedorenko from Massachusetts Institute of Technology, entitled ``The language system in the human brain''.

There are many people we would like to thank for their significant contributions. EMNLP 2021 would not be possible without their support:

\begin{itemize}

\item Our General Chair, Marie-Francine Moens, who has led the whole organizing team, and helped with many of our decision processes;
\item 46 SACs who have helped us comprehensively throughout the entire review process, from recruiting ACs and reviewers, assigning papers, checking review quality, making recommendation on final paper decisions, suggesting presentation formats, to recommending best paper candidates; special thanks to Jesse Dodge, who advocated to set up the ``Efficient methods for NLP'' track, volunteered to serve as the SAC, and helped us update the Reproducibility Checklist to encourage authors to report the computational budget for the experiments in their paper;
\item 237 ACs who checked the initial submissions, led paper discussions, wrote meta reviews, ensured review quality, suggested best paper candidates, and recommended outstanding reviewers;
\item 3,112 reviewers, 369 secondary reviewers for reviewing papers and actively participating in paper discussions; special thanks to those who stepped in at the last minute to serve as emergency reviewers;
\item 35 Ethics Committee members, chaired by Margot Mieskes and Chris Potts, for their hard work to provide ethical reviews and meta-reviews for all papers with serious ethical issues, and ensure that all the conditionally accepted papers have addressed the ethical issues appropriately in a very tight schedule;
\item Best Paper Award Committee: Luke Zettlemoyer (chair), Raffaella Bernardi, Mikel L. Forcada, Pascale Fung, Jianfeng Gao, Min Yen KAN, Heng Ji, Mausam, and Ivan Titov, for selecting best papers and outstanding papers under a tight schedule.
\item Our postdoc and student assistants Fernando Alva-Manchego, Zichu Fei, Yiding Tan, Yongxin Zhang and Xingwu Hu, who helped with the initial reviewer assignment, anonymity, multiple submission and format checking;
\item Past *ACL PCs, including Trevor Cohn, Yulan He and Yang Liu (EMNLP 2021), Fei Xia, Wenjie Li, Roberto Navigli (ACL-IJCNLP 2021), and Anna Rumshisky, Luke Zettlemoyer and Dilek Hakkani-Tur (NAACL 2021) for all the useful guidance, tips and suggestions on the organization of NLP conferences;
\item ARR Editors-in-chief Pascale Fung, Goran Glavaš, Sebastian Riedel, Amanda Stent, and CTO Graham Neubig, for their support in running the first ARR pilot, and providing the code for reviewer COI detection and paper assignment;
\item Publication Chairs Loic Barrault, Greg Durrett and Yansong Feng, and Findings Chairs Gabriel Stanovsky and Tim Van de Cruys, for completing the final proceedings within a short period;
\item ACL Anthology Director Matt Post, for his help in the production of the conference proceedings;
\item TACL editors-in-chief Mark Johnson, Ani Nenkova, and Brian Roark, TACL Editorial Assistant Cindy Robinson, and CL Editor-in-Chief Hwee Tou Ng for coordinating TACL and CL presentations with us;
\item Workshop Chairs Parisa Kordjamshidi and Minlie Huang, for connecting Findings paper authors with workshop organizers for possible presentations.
\item Publicity Chairs Raffaella Bernardi and Preethi Jyothi, Website Chair Miryam de Lhoneux, and Website Support Mingxiao Li, who announced conference news on EMNLP Website and social media, collected feedback from the community, and disseminated EMNLP papers with potential public interests via media;
\item Rich Gerber at SoftConf, who set up the EMNLP conference site, and was always quick to respond to our emails and resolve any problems we encountered with the START system;
\item Sol Rosenberg, Daniel Luise and the whole Underline team, for creating the virtual site for the conference and helping put the hybrid program in place;
\item Priscilla Rasmussen and members of the Local Organizing Committee, for various discussions on organizing EMNLP, and making the local arrangements for a hybrid programme;
\item SIGDAT board members, Iryna Gurevych, Hang Li, Mona Diab and Chin-Yew Lin, for their guidance regarding various decisions;
\item 19,272 authors for submitting their work to EMNLP 2021.

\end{itemize}

Our deepest gratitude to all of you. We hope you will enjoy the hybrid conference experience.
\vspace{3em}

\noindent \textit{Xuanjing Huang}, Fudan University\\
\noindent \textit{Lucia Specia}, Imperial College London\\
\noindent \textit{Scott Wen-tau Yih}, Facebook

\noindent EMNLP 2021 Program Co-Chairs \\

\index{Huang, Xuanjing}
\index{Specia, Lucia}
\index{Wen-tau Yih, Scott}
